\documentclass[10pt,a4paper,oneside]{article}
\usepackage[utf8]{inputenc}
\usepackage[english,russian]{babel}
\usepackage{amsmath}
\usepackage{amsthm}
\usepackage{amssymb}
\usepackage{enumerate}
\usepackage{stmaryrd}
\usepackage{cmll}
\usepackage{mathrsfs}
\usepackage[left=2cm,right=2cm,top=2cm,bottom=2cm,bindingoffset=0cm]{geometry}
\usepackage{proof}
\usepackage{tikz}
\usepackage{multicol}
\usepackage{mathabx}
\usepackage{comment}
\usepackage{hyperref}
\usepackage[utf8]{inputenc}
\usepackage[english,russian]{babel}
\usepackage{amssymb}
\usepackage{stmaryrd}
\usepackage{cmll}
\usepackage{xcolor}
\usepackage{proof}
\usetikzlibrary{hobby,fit,backgrounds,calc,shapes.geometric,patterns}
\usetikzlibrary{patterns}
\usepackage[normalem]{ulem}

\newcommand\doubleplus{+\kern-1.3ex+\kern0.8ex}
\newcommand\mdoubleplus{\ensuremath{\mathbin{+\mkern-10mu+}}}

\begin{document}

\section{Исчисление предикатов}
\subsection{Исчисление предикатов}
\subsubsection{Язык исчисления предикатов}
\begin{enumerate}
\item Два типа: предметные и логические выражения.
\item Предметные выражения: метапеременная {\color{blue}$\theta$}.
\begin{enumerate}
\item Предметные переменные: {\color{blue}$a$}, {\color{blue}$b$}, {\color{blue}$c$}, \dots, метапеременные {\color{blue}$x$}, {\color{blue}$y$}.
\item Функциональные выражения: {\color{blue}$f(\theta_1,\dots,\theta_n)$}, метапеременные {\color{blue}$f$}, {\color{blue}$g$}, \dots\\
%Имена разнообразны: {\color{blue}$(\theta_1+\theta_2)$}, {\color{blue}$0$} и т.п.
\item Примеры: %{\color{blue}$(x+1)^2$}. \\
% Раскроем сокращения: {\color{blue}$(\theta_0+\theta_1)\equiv p(\theta_0,\theta_1)$},
%   {\color{blue}$\theta_1^2\equiv q(\theta_1)$}, {\color{blue}$1\equiv r$}, {\color{blue}$2\equiv s$}. \\
  {\color{blue}$r$},
  {\color{blue}$q(p(x,s),r)$}.
\end{enumerate}
\item Логические выражения: метапеременные {\color{blue}$\alpha$}, {\color{blue}$\beta$}, {\color{blue}$\gamma$}, \dots
\begin{enumerate}
\item Предикатные выражения: {\color{blue}$P(\theta_1,\dots,\theta_n)$}, метапеременная {\color{blue}$P$}.\\
Имена: {\color{blue}$A$}, {\color{blue}$B$}, {\color{blue}$C$}, \dots  %также {\color{blue}$(\theta_1=\theta_2)$} и т.п.
\item Связки: {\color{blue}$(\varphi\vee\psi)$}, {\color{blue}$(\varphi\with\psi)$}, {\color{blue}$(\varphi\rightarrow\psi)$}, 
   {\color{blue}$(\neg\varphi)$}.
\item Кванторы: {\color{blue}$(\forall x.\varphi)$} и {\color{blue}$(\exists x.\varphi)$}.
\end{enumerate}
\end{enumerate}

\subsubsection{Сокращения записи, метаязык}

\begin{enumerate}
\item Метапеременные:
\begin{enumerate}
\item $\color{blue}\psi$, $\color{blue}\phi$, $\color{blue}\pi$, \dots --- формулы
\item $\color{blue}P$, $\color{blue}Q$, \dots --- предикатные символы
\item $\color{blue}\theta$, \dots --- термы
\item $\color{blue}f$, $\color{blue}g$, \dots --- функциональные символы
\item $\color{blue}x$, $\color{blue}y$, \dots --- предметные переменные
\end{enumerate}

\item Скобки --- как в И.В.; квантор --- жадный:
\begin{center}${\color{blue}(\forall a.} \underbrace{{\color{blue}A \vee B \vee C \rightarrow \exists b.}
                    \underbrace{\color{blue}D \with \neg E}_{\exists b.\dots}}_{\forall a.\dots} \color{blue}) \with F$\end{center}

\item Дополнительные обозначения при необходимости:
\begin{enumerate}
\item $\color{blue}(\theta_1 = \theta_2)$ вместо $\color{blue}E(\theta_1,\theta_2)$
\item $\color{blue}(\theta_1 + \theta_2)$ вместо $\color{blue}p(\theta_1,\theta_2)$
\item $\color{blue}0$ вместо $\color{blue}z$
\item \dots
\end{enumerate}

\end{enumerate}

\subsubsection{Оценка исчисления предикатов}
{\bf Определение.} Оценка --- упорядоченная четвёрка $\langle D, F, P, E \rangle$, где:

\begin{enumerate}
\item $D$ --- предметное множество;
\item $F$ --- оценка для функциональных символов; пусть $f_n$ --- $n$-местный функциональный символ:
 $$F_{f_n}: D^n \rightarrow D$$

\item $P$ --- оценка для предикатных символов; пусть $T_n$ --- $n$-местный предикатный символ:
 $$P_{T_n}: D^n \rightarrow V\quad\quad\quad V = \{\text{И}, \text{Л}\}$$

\item $E$ --- оценка для предметных переменных.
 $$E(x) \in D$$
\end{enumerate}

\subsubsection{Оценка формулы}

Запись и сокращения записи подобны исчислению высказываний: $$\llbracket \phi \rrbracket \in V,\quad
      \llbracket Q(x,f(x))\vee R\rrbracket^{x := 1, f(t) := t^2, R := \text{И}} = \text{И}$$

\begin{enumerate}
\item Правила для связок $\vee$, $\with$, $\neg$, $\rightarrow$ остаются прежние;
\item $\llbracket f_n (\theta_1, \theta_2, \dots, \theta_n) \rrbracket = F_{f_n} (\llbracket\theta_1\rrbracket,
          \llbracket\theta_2\rrbracket, \dots, \llbracket\theta_n\rrbracket)$
\item $\llbracket P_n (\theta_1, \theta_2, \dots, \theta_n) \rrbracket = P_{T_n} (\llbracket\theta_1\rrbracket,
          \llbracket\theta_2\rrbracket, \dots, \llbracket\theta_n\rrbracket)$
\item $$\llbracket \forall x.\phi \rrbracket = \left\{\begin{array}{ll}
   \text{И}, & \text{если } \llbracket\phi\rrbracket^{x := t} = \text{И}\text{ при всех } t \in D\\
   \text{Л}, & \text{если найдётся } t \in D, \text{ что } \llbracket\phi\rrbracket^{x := t} = \text{Л}
  \end{array}\right.$$
\item $$\llbracket \exists x.\phi \rrbracket = \left\{\begin{array}{ll}
   \text{И}, & \text{если найдётся } t \in D, \text{ что } \llbracket\phi\rrbracket^{x := t} = \text{И}\\
   \text{Л}, & \text{если } \llbracket\phi\rrbracket^{x := t} = \text{Л}\text{ при всех } t \in D
  \end{array}\right.$$
\end{enumerate}

\subsubsection{Аксиомы и правила вывода.}
Рассмотрим язык исчисления предикатов. Возьмём все схемы аксиом классического исчисления высказываний и добавим ещё две схемы аксиом 
(здесь везде $\theta$ свободен для подстановки вместо $x$ в $\varphi$):

\begin{tabular}{ll}
11. & $(\forall x.\varphi) \rightarrow \varphi[x:=\theta]$\\
12. & $\varphi[x:=\theta] \rightarrow \exists x.\varphi$ 
\end{tabular}

Добавим ещё два правила вывода (здесь везде $x$ не входит свободно в $\varphi$):
$$\infer[\text{Правило для }\forall]{\varphi\rightarrow\forall x.\psi}{\varphi\rightarrow\psi}$$
$$\infer[\text{Правило для }\exists]{(\exists x.\psi)\rightarrow\varphi}{\psi\rightarrow\varphi}$$

\subsection{Общезначимость, следование, выводимость}
\subsubsection{Общезначимость}
{\bf Определение.} Формула исчисления предикатов общезначима, если истинна при любой оценке:
$$\models\phi$$
То есть истинна при любых $D$, $F$, $P$ и $E$.

\subsubsection{Выводимость}


{\bf Определение.} Доказуемость, выводимость, полнота, корректность --- аналогично исчислению высказываний.

\subsubsection{Следование}
{\bf Определение.} $\gamma_1,\gamma_2,\dots,\gamma_n\models\alpha$, если выполнено два условия:
\begin{enumerate}
\item $\alpha$ выполнено всегда, когда выполнено $\gamma_1,\gamma_2,\dots,\gamma_n$;
\item $\alpha$ не использует кванторов по переменным, входящим свободно в $\gamma_1,\gamma_2,\dots,\gamma_n$.
\end{enumerate}

\subsection{Теорема о дедукции для исчисления предикатов}
\subsubsection{Теорема}
{\bf Теорема.} Если $\Gamma\vdash\alpha\rightarrow\beta$, то $\Gamma,\alpha\vdash\beta$.
Если $\Gamma,\alpha\vdash\beta$ и в доказательстве не применяются правила для кванторов 
по свободным переменным из $\alpha$, то $\Gamma\vdash\alpha\rightarrow\beta$.

\noindent {\bf Теорема.} Если $\Gamma\vdash\alpha$ и в доказательстве не используются кванторы по свободным
переменным из $\Gamma$, то $\Gamma\models\alpha$

\subsection{Корректность}
\subsubsection{Теорема}

{\bf Теорема.} Если $\Gamma \vdash \alpha$ и в доказательстве не используются кванторы по свободным переменным из $FV(\Gamma)$, то $\Gamma \models \alpha$


\section{Непротиворечивое множество формул}
\subsection{Непротиворечивое множество формул}
\subsubsection{Определение}
{\bf Определение.} $\Gamma$ --- \emph{непротиворечивое множество формул},
если $\Gamma\not\vdash\alpha\with\neg\alpha$ для любого $\alpha$

\subsubsection{Примеры}
\begin{enumerate}
\item непротиворечиво: \begin{itemize}
\item $\Gamma = \{A \rightarrow B \rightarrow A\}$
\item $\Gamma = \{P(x,y)\rightarrow\neg P(x,y), \forall x.\forall y.\neg P(x,y)\}$;
\end{itemize}
\item противоречиво: \begin{itemize}
\item $\Gamma = \{P\rightarrow\neg P, \neg P \rightarrow P\}$

так как
$P\rightarrow\neg P, \neg P \rightarrow P \ \vdash\  \neg P \with \neg\neg P$
\end{itemize}
\item пусть $D = \mathbb{Z}$ и $P(x) \equiv (x > 0)$, аналогом для этой модели
будет $\Gamma = \{P(1), P(2), P(3), \dots\}$
\end{enumerate}

\subsection{Полное непротиворечивое множество формул}
{\bf Определение.} $\Gamma$ --- \emph{полное} непротиворечивое множество замкнутых бескванторных формул,
если:
\begin{enumerate}\item $\Gamma$ содержит только замкнутые бескванторные формулы;
\item если $\alpha$ --- некоторая замкнутая бескванторная формула, то $\alpha\in\Gamma$ или $\neg\alpha\in\Gamma$.
\end{enumerate}

\noindent {\bf Определение.} $\Gamma$ --- \emph{полное} непротиворечивое множество замкнутых формул, если:
\begin{enumerate}\item $\Gamma$ содержит только замкнутые формулы;
\item если $\alpha$ --- некоторая замкнутая формула, то $\alpha \in \Gamma$, или $\neg\alpha \in \Gamma$.
\end{enumerate}

\noindent {\bf Теорема.} Пусть $\Gamma$ --- непротиворечивое множество замкнутых (бескванторных) формул. Тогда, какова бы ни была
замкнутая (бескванторная) формула $\varphi$, хотя бы $\Gamma \cup \{\varphi\}$ или $\Gamma \cup \{\neg\varphi\}$ ---
непротиворечиво.

\subsection{Доказательство существования моделей у непротиворечивых множеств формул 
в бескванторном исчислении предикатов.}
\subsubsection{Модель для множества формул}
\noindent {\bf Определение.} Моделью для множества формул $F$ назовём такую модель $\mathcal{M}$, что
    при всяком $\varphi \in F$ выполнено $\llbracket\varphi\rrbracket_\mathcal{M} = \text{И}$
    
\noindent Альтернативное обозначение: $\mathcal{M}\models\varphi$.

\subsubsection{Теорема}
\noindent {\bf Теорема.} Любое непротиворечивое множество замкнутых бескванторных формул имеет модель.    

\subsubsection{Доказательство теоремы о существовании модели}
{\bf Лемма.} Пусть $\varphi$ --- бескванторная формула, тогда $\mathcal{M}\models\varphi$ тогда и только тогда, когда $\varphi\in M$


\noindent {\bf Докозательство теоремы.}

Пусть $M$ --- непротиворечивое множество замкнутых бескванторных формул.

По теореме о пополнении существует $M'$ --- полное непротиворечивое множество замкнутых бескванторных формул,
что $M \subseteq M'$.

По лемме $M'$ имеет модель, эта модель подойдёт для $M$.

\subsection{Теорема Гёделя о полноте исчисления предикатов}
\subsubsection{Теорема}
{\bf Теорема.} Если $M$ --- замкнутое непротиворечивое множество формул, то оно имеет модель.

\subsection{Полнота исчисления предикатов}
\subsubsection{Следствие}
{\bf Следствие (из теоремы Гёделя о полноте).}
Исчисление предикатов полно.

\section{Машина Тьюринга. Задача об останове, её неразрешимость. Неразрешимость исчисления предикатов.}

\subsection{Машина Тьюринга.}
\noindent {\bf Определение.} Машина Тьюринга:
\begin{enumerate}
\item Внешний алфавит $q_1, \dots, q_n$, выделенный символ-заполнитель $q_\varepsilon$
\item Внутренний алфавит (состояний) $s_1, \dots, s_k$; $s_s$ --- начальное, $s_f$ --- допускающее, $s_r$ --- отвергающее.
\item Таблица переходов $\langle k, s \rangle \Rightarrow \langle k', s', \leftrightarrow \rangle$
\end{enumerate}

\noindent {\bf Определение.} Состояние машины Тьюринга:
\begin{enumerate}
\item Бесконечная лента с символом-заполнителем $q_\varepsilon$, текст конечной длины.
\item Головка над определённым символом.
\item Символ состояния (состояние в узком смысле) --- символ внутреннего алфавита.
\end{enumerate}

\subsection{Задача об останове, её неразрешимость.}
\subsubsection{Разрешимость.}
\noindent {\bf Определение.} Язык --- множество строк

\noindent {\bf Определение.} Язык $L$ разрешим, если существует машина Тьюринга, которая для любого слова $w$ переходит в допускающее состояние, если $w \in L$,
и в отвергающее, если $w \notin L$.

\subsubsection{Неразрешимость задачи останова.}
\noindent {\bf Определение.} Рассмотрим все возможные описания машин Тьюринга. Составим упорядоченные пары: описание машины Тьюринга и входная строка.
Из них выделим язык останавливающихся на данном входе машин Тьюринга.

\noindent{\bf Теорема.} Язык всех останавливающихся машин Тьюринга неразрешим.

\subsection{Неразрешимость исчисления предикатов: доказательство}
\noindent{\bf Теорема.} Язык всех доказуемых формул исчисления предикатов неразрешим
Т.е. нет машины Тьюринга, которая бы по любой формуле $\alpha$ определяла, доказуема ли она.

\noindent{\bf Доказательство.}  Пусть существует машина Тьюринга, разрешающая любую формулу.
На её основе тогда несложно построить некоторую машину Тьюринга, перестраивающую любую машину $S$ (с допускающим состоянием $s_f$ и входом $y$) 
в её ограничения $C$ и разрешающую формулу ИП $C \rightarrow \exists w_l.\exists w_r.F_{S,y}(w_l,w_r,s_f)$. 
Эта машина разрешит задачу останова.

\section{Порядок теории (0, 1, 2). Теории первого порядка. Аксиоматика Пеано. Арифметические операции. Формальная арифметика.}
\subsection{Порядок теории (0, 1, 2).}

\subsubsection{Теория первого порядка}
\noindent {\bf Определение.} 
Теорией первого порядка назовём исчисление предикатов с дополнительными (<<нелогическими>>
или <<математическими>>):
\begin{enumerate}
\item предикатными и функциональными символами;
\item аксиомами.
\end{enumerate}

Сущности, взятые из исходного исчисления предикатов, назовём \emph{логическими}

\subsubsection{Порядок логики/теории}
\begin{tabular}{llll}
Порядок & Кванторы & Формализует суждения\dots & Пример\\\hline
нулевой & запрещены & об отдельных значениях & И.В.\\
первый & по предметным переменным & о множествах & И.П.\\
%    &   & $S = \{ t\ |\ \psi[x := t] \}$ \\
    &   \multicolumn{2}{l}{\color{olive}$\{2,3,5,7,\dots\} = \{ t\ |\ \forall p.\forall q.(p \ne 1 \with q \ne 1) \rightarrow (t \ne p\cdot q)\}$}\\
второй & по предикатным переменным & о множествах множеств & Типы\\
    &   \multicolumn{2}{l}{\color{olive}$S = \{ \{t\ |\ P(t)\}\ |\ \varphi[p := P] \}$}\\
 & \dots 
\end{tabular}

\subsection{Аксиоматика Пеано.}
\subsubsection{Натуральные числа: аксиоматика Пеано, 1889}\vspace{-0.5cm}
 $$\mathbb{N}: 1, 2, \dots \mbox{ или } \mathbb{N}_0: 0, 1, 2, \dots$$\vspace{-1cm}
\noindent {\bf Определение.} 
  $N$ (или, более точно, $\langle N, 0, (')\rangle$) \emph{соответствует} аксиоматике Пеано, 
  если следующее определено/выполнено:
  \begin{enumerate}
     \item Операция <<штрих>> $('): N \to N$, причём нет $a,b \in N$, что $a \ne b$, но $a' = b'$.
           
           Если $x = y'$, то $x$ назовём следующим за $y$, а $y$ --- предшествующим $x$.
     \item Константа $0 \in N$: нет $x \in N$, что $x' = 0$.
     \item Индукция. Каково бы ни было свойство (<<предикат>>) $P: N \to V$, если:
           \begin{enumerate}
           \item $P(0)$
           \item При любом $x\in N$ из $P(x)$ следует $P(x')$
           \end{enumerate}
           то при любом $x \in N$ выполнено $P(x)$.
  \end{enumerate}
Как построить? Например, в стиле алгебры Линденбаума:
\begin{enumerate}
\item $N$ --- язык, порождённый грамматикой $\nu ::= \texttt{0}\ |\ \nu \texttt{<<'>>}$
\item $0$ --- это $\text{<<0>>}$, $x'$ --- это $x \doubleplus \texttt{<<'>>}$
\end{enumerate}

\subsection{Арифметические операции.}

\subsubsection{Обозначения и определения}
\noindent {\bf Определение.}
$1 = 0'$, $2 = 0''$, $3 = 0'''$, $4 = 0''''$, $5 = 0'''''$, $6 = 0''''''$,
$7 = 0'''''''$, $8 = 0''''''''$, $9 = 0'''''''''$

\noindent {\bf Определение.}\vspace{-0.3cm}
$$a + b = \left\{ \begin{array}{ll} a, & \mbox{если } b = 0\\
                                    (a + c)', & \mbox{если } b = c'
                  \end{array}\right.$$
\vspace{-0.3cm}
Например, $$2 + 2 = 0'' + 0'' = (0'' + 0')' = ((0'' + 0)')' = ((0'')')' = 0'''' = 4$$\vspace{-0.3cm}

\noindent {\bf Определение.}\vspace{-0.3cm}
$$a \cdot b = \left\{ \begin{array}{ll} 0, & \mbox{если } b = 0\\
                                    a \cdot c + a, & \mbox{если } b = c'
                  \end{array}\right.$$
                  
\subsubsection{Коммутативность сложения.}
\noindent{\bf Теорема.} 
$a + b = b + a$

\subsection{Формальная арифметика.}
\noindent {\bf Определение.} 
Формальная арифметика --- теория первого порядка, со следующими добавленными нелогическими \dots
\begin{enumerate}
\item двухместными функциональными символами $(+)$, $(\cdot)$; одноместным функциональным символом $(')$, 
нульместным функциональным символом $0$;
\item двухместным предикатным символом $(=)$;
\item восемью нелогическими \emph{аксиомами}:\vspace{0.1cm}
\begin{tabular}{ll}
(A1) $a=b \to a=c \to b=c$             &(A5) $a+0 = a$                     \\
(A2) $a=b \to a'=b'$                   &(A6) $a+b' = (a+b)'$               \\
(A3) $a'=b' \to a=b$                   &(A7) $a\cdot 0 = 0$                \\
(A4) $\neg a' = 0$                     &(A8) $a\cdot b' = a \cdot b + a$
\end{tabular}
\item нелогической схемой аксиом индукции $\psi[x:=0]\with(\forall x.\psi\to \psi[x:=x'])\to \psi$ с метапеременными $x$ и $\psi$.
\end{enumerate}

\section{Примитивно-рекурсивные и рекурсивные функции. 
функций вычисления простых чисел. Частичный логарифм.
Выразимость отношений и представимость функций в формальной арифметике. Характеристические функции.
Функция Аккермана.}

\subsection{Примитивно-рекурсивные функции}

\noindent {\bf Определение (Примитивы Z, N, U, S).}
\begin{enumerate}
\item Примитив <<Ноль>> ($Z$) \vspace{-0.3cm}
$$Z: \mathbb{N}_0\to\mathbb{N}_0,\ \ \ \ \ Z(x_1) = 0$$\vspace{-0.5cm}
\item Примитив <<Инкремент>> ($N$) \vspace{-0.3cm}
$$N: \mathbb{N}_0\to\mathbb{N}_0,\ \ \ \ \ N(x_1) = x_1+1$$\vspace{-0.5cm}
\item Примитив <<Проекция>> ($U$) — семейство функций; пусть $k,n \in \mathbb{N}_0, k \le n$\vspace{-0.2cm}
$$U^k_n: \mathbb{N}^n_0 \to \mathbb{N}_0,\ \ \ \ \ U^k_n(\overrightarrow{x}) = x_k$$\vspace{-0.5cm}
\item Примитив <<Подстановка>> ($S$) --- семейство функций; пусть $g: \mathbb{N}^k_0 \to \mathbb{N}_0,\ \ f_1,\dots,f_k: \mathbb{N}^n_0 \to \mathbb{N}_0$
$$S\langle g,f_1,f_2,\dots,f_k \rangle (\overrightarrow{x}) = g(f_1(\overrightarrow{x}),\dots,f_k(\overrightarrow{x}))$$
\end{enumerate}

\subsection{Примитивная рекурсия}
\subsubsection{Определения.}
\noindent {\bf Определение.} [примитив <<примитивная рекурсия>>, $R$]
Пусть $f: \mathbb{N}^n_0\to\mathbb{N}_0$ и $g: \mathbb{N}^{n+2}_0 \to\mathbb{N}_0$.
Тогда $R\langle f,g\rangle: \mathbb{N}^{n+1}_0\to\mathbb{N}_0$, причём

$$R\langle f,g\rangle(\overrightarrow{x},y)=
 \left\{\begin{array}{ll} 
  f(\overrightarrow{x}), &y=0\\
  g(\overrightarrow{x},y-1,R\langle f,g\rangle (\overrightarrow{x},y-1)), &y > 0
\end{array}\right.$$

\noindent {\bf Определение.} Функция $f$ --- примитивно-рекурсивна, если может быть
выражена как композиция примитивов $Z$, $N$, $U$, $S$ и $R$.

\noindent {\bf Теорема.} 
$f(x) = x+2$ примитивно-рекурсивна


\noindent {\bf Лемма.} 
$f(a,b) = a+b$ примитивно-рекурсивна

\subsubsection {Какие функции примитивно-рекурсивные?}
\begin{enumerate}
\item Сложение, вычитание
\item Умножение, деление
\item Вычисление простых чисел
\item Неформально: все функции, вычисляемые конечным числом вложенных циклов \verb!for!:

\begin{verbatim}
for (int i1 = 0; i1 < g1(x1...xn); i1++) {
    for (int i2 = 0; i2 < g2(x1...xn,i1); i2++) {
        ...
           for (int ik = 0; ik < gk(x1...xn,i1,i2...); ik++) {
               // выражение без циклов
           }
        ...
    }
}
\end{verbatim}
\end{enumerate}

\subsection{Частичный логарифм.}

$plog_k(p^n \cdot m^t \cdot k^a) = a$

\subsection{Выразимость отношений и представимость функций в формальной арифметике.}
\noindent {\bf Теорема.} Любая рекурсивная функция представима в Ф.А.

\noindent {\bf Теорема.} Любая представимая в Ф.А. функция рекурсивна.

\noindent {\bf Определение.} 
Будем говорить, что функция $f: \mathbb{N}^n_0\to\mathbb{N}_0$ представима в ФА, 
если существует формула $\varphi$, что:
\begin{enumerate}
\item если $f(a_1,\dots,a_n) = u$, то $\vdash \varphi(\overline{a_1},\dots,\overline{a_n},\overline{u})$
\item если $f(a_1,\dots,a_n) \ne u$, то $\vdash \neg\varphi(\overline{a_1},\dots,\overline{a_n},\overline{u})$
\item для всех $a_i \in \mathbb{N}_0$ выполнено $\vdash (\exists x.\varphi(\overline{a_1},\dots,\overline{a_n},x)) \with 
   (\forall p.\forall q.\varphi(\overline{a_1},\dots,\overline{a_n},p)\with \varphi(\overline{a_1},\dots,\overline{a_n},q)\rightarrow p=q)$
\end{enumerate}

\noindent {\bf Определение.} 
Будем говорить, что отношение $R\subseteq \mathbb{N}^n_0$ выразимо в ФА, 
если существует формула $\rho$, что:
\begin{enumerate}
\item если $\langle a_1,\dots,a_n \rangle \in R$, то $\vdash \rho(\overline{a_1},\dots,\overline{a_n})$
\item если $\langle a_1,\dots,a_n \rangle \notin R$, то $\vdash \neg\rho(\overline{a_1},\dots,\overline{a_n})$
\end{enumerate}


\noindent {\bf Теорема.}  отношение <<равно>> выразимо в Ф.А.:
$R = \{ \langle x,x \rangle\ |\ x \in \mathbb{N}_0 \}$

\subsection{Характеристические функции.}
Характеристическая функция арифметического отношения $R$ - это функция $C_R(x_1,...,x_n) = \begin{cases}0 \  R(x_1,...,x_n) \\ 1 \  R(x_1,...,x_n) \  \text{неверно} \end{cases}$

\subsection{Функция Аккермана.}
\noindent {\bf Определение.} 
Функция Аккермана:
$$A(m,n) = \left\{\begin{array}{ll}
  n+1,&m = 0\\
  A(m-1,1),&m > 0, n = 0\\
  A(m-1,A(m,n-1)),&m > 0, n > 0
\end{array}\right.$$

\section{Бета-функция Гёделя. 
Гёделева нумерация. Рекурсивность представимых в формальной арифметике функций.}

\subsection{Бета-функция Гёделя.}
Задача: закодировать последовательность натуральных чисел произвольной длины.
\noindent {\bf Определение.} $\beta$-функция Гёделя: $\mathcal{\beta}(b,c,i) := b \% (1 + (i+1) \cdot c)$\\
Здесь (\%) --- остаток от деления.

\noindent {\bf Теорема.} $\beta$-функция Гёделя представима в Ф.А. формулой
$$\hat{\beta}(b,c,i,d) := \exists q.(b = q \cdot (1 + c \cdot (i+1)) + d) \& (d < 1 + c \cdot (i+1))$$

Деление $b$ на $x$ с остатком: найдутся частное $(q)$ и остаток $(d)$, что
$b = q\cdot x + d$ и $0 \le d < x$.

\noindent {\bf Теорема.} Если $a_0, \dots, a_n \in \mathbb{N}_0$, то найдутся такие $b,c \in \mathbb{N}_0$, что
$a_i = \beta(b,c,i)$

\subsection{Гёделева нумерация.}
\begin{enumerate}
\item Отдельный символ.\vspace{0.2cm}

\begin{tabular}{lc|lc||cll}
Номер & Символ & Номер & Символ & Имя & $k,n$ & Гёделев номер\\
3 & ( &               17 & $\&$ &  0 & $0,0$ & $27 + 6$\\
5 & ) &               19 & $\forall$ & $(')$ & $0,1$ & $27 + 6 \cdot 3$\\
7 & , &               21 & $\exists$ & $(+)$ & $0,2$ & $27 + 6 \cdot 9$\\
9 & . &               23 & $\vdash$ & $(\cdot)$ & $1,2$ & $27 + 6 \cdot 2 \cdot 9$\\
11 & $\neg$ &         $25 + 6\cdot k$ & $x_k$ & $(=)$ & $0,2$ & $29 + 6 \cdot 9$\\
13 & $\rightarrow$ &  $27 + 6\cdot 2^k \cdot 3^n$ & $f_k^n$\\
15 & $\vee$ &         $29 + 6\cdot 2^k \cdot 3^n$ & $P_k^n$ 
\end{tabular}

\item Формула. $\phi \equiv s_0s_1\dots s_{n-1}$. Гёделев номер: $\ulcorner\phi\urcorner = 2^{\ulcorner s_0\urcorner}\cdot 3^{\ulcorner s_1\urcorner} 
\cdot \dots \cdot p_{n-1}^{\ulcorner s_{n-1}\urcorner}$.

\item Доказательство. $\Pi = \delta_0\delta_1\dots\delta_{k-1}$, его гёделев номер: $\ulcorner\Pi\urcorner =
2^{\ulcorner \delta_0\urcorner}\cdot 3^{\ulcorner \delta_1\urcorner} \cdot \dots \cdot p_{k-1}^{\ulcorner \delta_{k-1}\urcorner}$
\end{enumerate}

\subsection{Рекурсивность представимых в формальной арифметике функций.}
\subsubsection{Представимость рекурсивных функций в Ф.А.}
\noindent {\bf Теорема.} Пусть функция $f:\mathbb{N}^{n+1}_0 \to \mathbb{N}_0$ представима в Ф.А.
формулой $\varphi(x_1,\dots,x_{n},y,r)$. Тогда примитив $M\langle f\rangle$ представим в Ф.А. 
формулой $$\mu(x_1,\dots,x_n,y) := \varphi(x_1,\dots,x_n,y,0) \with \forall u.u < y \to \neg\varphi(x_1,\dots,x_n,u,0)$$

\noindent {\bf Теорема.} Если $f$ --- рекурсивная функция, то она представима
в Ф.А.


\subsubsection{Рекурсивность представимых функций в Ф.А.}
Фиксируем $f$ и $x_1, x_2, \dots, x_n$. Обозначим $y = f(x_1,x_2,\dots,x_n)$.
По представимости нам известна $\varphi$, что $\vdash \varphi(\overline{x_1},\overline{x_2},\dots,\overline{x_n},\overline{y})$.
Давайте просто переберём все результаты и доказательства!

\begin{enumerate}
\item Закодируем доказательства натуральными числами.
\item Напишем рекурсивную функцию, проверяющую доказательства на корректность.
\item Параллельный перебор значений и доказательств: $s = 2^y \cdot 3^p$. Переберём все $s$, по $s$ получим $y$ и $p$.
Проверим, что $p$ --- код доказательства $\vdash \varphi(\overline{x_1},\overline{x_2},\dots,\overline{x_n},\overline{y})$.
\end{enumerate}

\section{Непротиворечивость (эквивалентные определения), $\omega$-не\-про\-ти\-во\-ре\-чи\-вость. 
Первая теорема Гёделя о неполноте арифметики.
Формулировка первой теоремы Гёделя о неполноте арифметики в форме Россера. 
Синтаксическая и семантическая неполнота арифметики.
Неполнота расширений формальной арифметики.
Ослабленные варианты: арифметика Пресбургера, система Робинсона.}

\subsection{Непротиворечивость (эквивалентные определения), $\omega$-не\-про\-ти\-во\-ре\-чи\-вость.}
\noindent {\bf Определение.} Если для любой формулы $\phi(x)$ из $\vdash\phi(0)$, $\vdash\phi(\overline{1})$,
$\vdash\phi(\overline{2})$, $\dots$ выполнено $\not\vdash\exists x.\neg\phi(x)$, 
то теория \emph{омега-непротиворечива}.


\subsection{Первая теорема Гёделя о неполноте арифметики.}

\noindent {\bf Теорема.} {Первая теорема Гёделя о неполноте арифметики}
\begin{itemize}
\item Если формальная арифметика непротиворечива, то $\not\vdash\sigma(\overline{\ulcorner\sigma\urcorner})$.
\item Если формальная арифметика $\omega$-непротиворечива, то $\not\vdash\neg\sigma(\overline{\ulcorner\sigma\urcorner})$.
\end{itemize}

\subsection{Формулировка первой теоремы Гёделя о неполноте арифметики в форме Россера.}
\noindent {\bf Определение.} $\theta_1\le\theta_2 \equiv \exists p.p+\theta_1=\theta_2\quad\quad\theta_1<\theta_2\equiv\theta_1\le\theta_2\with\neg\theta_1=\theta_2$

\noindent {\bf Определение.} Пусть $\langle \ulcorner\xi\urcorner,p\rangle \in W_2$, если $\vdash\neg\xi(\overline{\ulcorner\xi\urcorner})$.
Пусть $\omega_2$ выражает $W_2$ в формальной арифметике.

\noindent {\bf Теорема.} Рассмотрим $\rho(x_1) = \forall p.\omega_1(x_1,p)\rightarrow\exists q.q \le p \with \omega_2(x_2,q)$.
Тогда $\not\vdash\rho(\overline{\ulcorner\rho\urcorner})$ и $\not\vdash\neg\rho(\overline{\ulcorner\rho\urcorner})$.
$\rho(\overline{\ulcorner\rho\urcorner})$: <<Меня легче опровергнуть, чем доказать>>


\subsection{Синтаксическая и семантическая неполнота арифметики.}

\noindent {\bf Определение.} \emph{Семантически} полная теория --- теория, в которой любая общезначимая формула доказуема.\\
\emph{Синтаксически} полная теория --- теория, в которой для каждой формулы $\alpha$ выполнено $\vdash\alpha$ или $\vdash\neg\alpha$.

\noindent {\bf Теорема.} Формальная арифметика с классической моделью семантически неполна.

\subsection{Неполнота расширений формальной арифметики.}

\noindent {\bf Определение.} Теория $\mathcal{S}$ --- расширение теории $\mathcal{T}$, если
из $\vdash_\mathcal{T} \alpha$ следует $\vdash_\mathcal{S} \alpha$


\noindent {\bf Определение.} Теория $\mathcal{S}$ --- рекурсивно-аксиоматизируемая, если найдётся теория $\mathcal{S'}$ с тем же языком, что:
\begin{enumerate}
\item $\vdash_\mathcal{S} \alpha$ тогда и только тогда, когда $\vdash_\mathcal{S'} \alpha$;
\item Множество аксиом теории $\mathcal{S'}$ рекурсивно.
\end{enumerate}

\noindent {\bf Теорема.} Если $\mathcal{S}$ --- непротиворечивое рекурсивно-аксиоматизируемое расширение формальной арифметики, то
в ней можно доказать аналоги теорем Гёделя о неполноте арифметики.

\subsection{Ослабленные варианты: арифметика Пресбургера, система Робинсона.}

\subsubsection{Арифметика Пресбургера}
\noindent {\bf Определение.} Теория первого порядка, использующая нелогические функциональные символы $0$, $1$, $(+)$, нелогический
предикатный символ $(=)$ и следующие нелогические аксиомы, называется арифметикой Пресбургера.

$$\begin{array}{l}
\neg (0 = x + 1) \\
x + 1 = y + 1 \rightarrow x = y\\
x + 0 = x \\
x + (y + 1) = (x + y) + 1\\
(\varphi(0) \with \forall x.\varphi(x) \rightarrow \varphi(x+1)) \rightarrow \forall y.\varphi(y)
\end{array}$$

\noindent {\bf Теорема.} Арифметика Пресбургера разрешима и синтаксически и семантически полна.

\subsubsection{Сужение: система Робинсона}
\noindent {\bf Определение.} Теория первого порядка, использующая нелогические функциональные символы $0$, $(+)$ и $(\cdot)$, нелогический
предикатный символ $(=)$ и следующие нелогические аксиомы, называется системой Робинсона.

\vspace{-0.4cm}
$$\begin{array}{ll}
a = a & a = b \rightarrow b = a \\
a = b \rightarrow b = c \rightarrow a = c & a = b \rightarrow a' = b' \\
a' = b' \rightarrow a = b & \neg 0 = a' \\
a = b \rightarrow a + c = b + c \with c + a = c + b & a = b \rightarrow a \cdot c = b \cdot c \with c \cdot a = c \cdot b \\
\neg a = 0 \rightarrow \exists b. a = b' & a + 0 = a\\
a + b' = (a + b)' & a \cdot 0 = 0 \\
a \cdot b' = a \cdot b + a 
\end{array}$$


\vspace{-0.3cm}
Система Робинсона неполна: аксиомы --- в точности утверждения, необходимые для доказательства теорем Гёделя.
Система Робинсона не имеет схем аксиом.

\section{Вторая теорема Гёделя о неполноте арифметики, $Consis$. 
Лемма об автоссылках. Условия Гильберта-Бернайса-Лёба. Неразрешимость формальной арифметики. Теорема Тарского о невыразимости истины.}

\subsection{Вторая теорема Гёделя о неполноте арифметики, Consis.}
\subsubsection{Consis}
\noindent {\bf Лемма.} 
$\vdash 1=0$ тогда и только тогда, когда $\vdash\alpha$ при любом $\alpha$.


\noindent {\bf Определение.} 
Обозначим за $\psi(x,p)$ формулу, выражающую в формальной арифметике рекурсивное
отношение Proof: $\langle \ulcorner\xi\urcorner,p\rangle \in \text{Proof}$, если $p$ --- гёделев номер
доказательства $\xi$. \\
Обозначим $\pi(x)\equiv\exists p.\psi(x,p)$

\noindent {\bf Определение.} Формулой Consis назовём формулу
$\neg \pi(\overline{\ulcorner 1=0 \urcorner})$

\noindent Неформальный смысл: <<формальная арифметика непротиворечива>>

\subsubsection{Вторая теорема Гёделя о неполноте арифметики}
\noindent {\bf Теорема.} Если Consis доказуем, то формальная арифметика противоречива.

\subsection{Лемма об автоссылках.}
\noindent {\bf Лемма.} Лемма об автоссылках. Для любой формулы $\phi(x_1)$ можно построить 
такую замкнутую формулу $\alpha$ (не использующую неаксиоматических предикатных
и функциональных символов), что $\vdash \phi(\overline{\ulcorner\alpha\urcorner}) \leftrightarrow \alpha$.

\subsection{Условия Гильберта-Бернайса-Лёба.}
\noindent {\bf Определение.} 
Будем говорить, что формула $\psi$, выражающая отношение Proof, 
формула $\pi$ и формула Consis соответствуют
условиям Гильберта-Бернайса-Лёба, если следующие условия выполнены для любой формулы $\alpha$:

\begin{enumerate}
\item $\vdash \alpha$ влечет $\vdash \pi(\overline{\ulcorner\alpha\urcorner})$
\item $\vdash \pi (\overline{\ulcorner\alpha\urcorner}) \rightarrow \pi(\overline{\ulcorner\pi(\overline{\ulcorner\alpha\urcorner})\urcorner})$
\item $\vdash \pi (\overline{\ulcorner\alpha\rightarrow \beta\urcorner}) \rightarrow \pi(\overline{\ulcorner\alpha\urcorner}) \rightarrow \pi(\overline{\ulcorner\beta\urcorner})$
\end{enumerate}


\subsection{Неразрешимость формальной арифметики.}
\noindent {\bf Теорема.}
Если формальная арифметика непротиворечива, то формальная арифметика неразрешима

\subsection{Теорема Тарского о невыразимости истины.}
\noindent {\bf Теорема.}
Не существует формулы $\varphi(x)$, что $\llbracket \varphi(x) \rrbracket = \text{И}$ (в стандартной интерпретации) тогда и только
тогда, когда $x \in \text{Tr}_\text{ФА}$.

\noindent Однако, если взять $D = \mathbb{R}$, истина становится выразима (алгоритм Тарского).

\section{ Лямбда-исчисление. Пред-лямбда-термы и лямбда-термы. Альфа-эквивалентность, бета-редукция
и бета-эквивалентность. Теорема Чёрча-Россера. 
Комбинатор неподвижной точки. Комбинаторный базис $SK$.
Истина и ложь. Чёрчевские нумералы. 
Натуральный вывод. Импликативный фрагмент интуиционистского исчисления высказываний.
Просто-типизированное лямбда исчисление. Изоморфизм Карри-Ховарда (высказывание, доказательство, импликация, конъюнкция, дизъюнкция, ложь).}

\subsection{Лямбда-исчисление.}
$$\Lambda ::= (\lambda x.\Lambda) | (\Lambda\ \Lambda) | x$$

Мета-язык: 
\begin{enumerate}
\item Мета-переменные:\begin{enumerate}
\item $A\dots Z$ --- мета-переменные для термов. 
\item $x,y,z$ --- мета-переменные для переменных. 
\end{enumerate}

\item Правила расстановки скобок аналогичны правилам для кванторов:
\begin{enumerate}
\item Лямбда-выражение ест всё до конца строки
\item Аппликация левоассоциативна
\end{enumerate}
\end{enumerate}

{\bf Примеры.}
\begin{enumerate}
\item $a\ b\ c\ (\lambda d.e\ f\ \lambda g.h)\ i \equiv \Big({\color{red}\Big(}((a\ b)\ c)\ {\color{blue}\Big(}\lambda d.((e\ f)\ (\lambda g.h)){\color{blue}\Big)}{\color{red}\Big)}\ i\Big)$
\item $0 := \lambda f.\lambda x.x;\quad(+1) := \lambda n.\lambda f.\lambda x.n\ f\ (f\ x);\quad(+2) := \lambda x.(+1)\ ((+1)\ x)$
\end{enumerate}


\subsection{Пред-лямбда-термы и лямбда-термы.}

\subsection{Альфа-эквивалентность, бета-редукция
и бета-эквивалентность.}
\subsubsection{Альфа-эквивалентность}
$$FV(A) = \left\{\begin{array}{ll} \{x\}, & A \equiv x\\
  FV(P)\cup FV(Q), & A \equiv P\ Q\\
  FV(P)\setminus\{x\}, & A \equiv \lambda x.P\end{array}\right.$$

\noindent {\bf Определение.} $A=_\alpha B$, если и только если выполнено одно из трёх:
\begin{enumerate}
\item $A \equiv x$, $B \equiv y$, $x \equiv y$;
\item $A \equiv P_a Q_a$, $B \equiv P_b Q_b$ и $P_a =_\alpha P_b$, $Q_a =_\alpha Q_b$;
\item $A \equiv (\lambda x.P)$, $B \equiv (\lambda y.Q)$, $P[x := t] =_\alpha Q[y := t]$, где $t$ не входит в $A$ и $B$.
\end{enumerate}

\subsubsection{Бета-редукция}
\noindent {\bf Определение.}  Терм вида $(\lambda x.P)\ Q$ --- бета-редекс.

\noindent {\bf Определение.}  $A \rightarrow_\beta B$, если:
\begin{enumerate}
\item $A \equiv (\lambda x.P)\ Q$, $B \equiv P\ [x := Q]$, при условии свободы для подстановки;
\item $A \equiv (P\ Q)$, $B \equiv (P'\ Q')$, при этом $P \rightarrow_\beta P'$ и $Q = Q'$, либо $P = P'$ и $Q \rightarrow_\beta Q'$;
\item $A \equiv (\lambda x.P)$, $B \equiv (\lambda x.P')$, и $P \rightarrow_\beta P'$.
\end{enumerate}

\subsubsection{Бета-эквивалентность}

\noindent {\bf Определение.}  $(=_\beta)$ --- транзитивное, рефлексивное и симметричное замыкание $(\rightarrow_\beta)$.

\subsection{Теорема Чёрча-Россера.}
\noindent {\bf Теорема (Чёрча-Россера).} Для любых термов $N$, $P$, $Q$, если $N \twoheadrightarrow_\beta P$, $N \twoheadrightarrow_\beta Q$,
и $P \ne Q$, то найдётся $T$: $P \twoheadrightarrow_\beta T$ и $Q \twoheadrightarrow_\beta T$.

\noindent {\bf Теорема.} Если у терма $N$ существует нормальная форма, то она единственна

\subsection{Комбинатор неподвижной точки.}

\noindent {\bf Теорема.} Для любого терма $N$ найдётся такой терм $R$, что $R =_\beta N\ R$.

$Y = \lambda f.(\lambda x.f\ (x\ x))\ (\lambda x.f\ (x\ x))$.

\subsection{Комбинаторный базис $SK$.}
\noindent {\bf Определение.}  Комбинатор --- лямбда-терм без свободных переменных

\noindent {\bf Определение.}  $S := \lambda x.\lambda y.\lambda z.x\ z\ (y\ z)$, $K := \lambda x.\lambda y.x$, $I := \lambda x.x$

\noindent {\bf Теорема.} Пусть $N$ --- некоторый замкнутый лямбда-терм. Тогда найдётся выражение $C$, состоящее из комбинаторов $S$,$K$, 
что $N =_\beta C$

\subsection{Истина и ложь.}

$$T = \lambda x.\lambda y.x$$

$$F = \lambda x.\lambda y.y$$


\subsection{Чёрчевские нумералы.}

$$f^{(n)}(x) = \left\{\begin{array}{ll}x, & n = 0\\f(f^{(n-1)}(x)), & n > 0\end{array}\right.$$

\noindent {\bf Определение.}  
Чёрчевский нумерал $\overline{n} = \lambda f.\lambda x.f^{(n)}(x)$


\subsection{Натуральный вывод.}

\begin{enumerate}
\item Формулы языка (секвенции) имеют вид: $\Gamma\vdash\alpha$.
Правила вывода: 
\begin{flushright}$\quad\quad\quad\infer[(\text{аннотация})]{\text{заключение}}{\text{посылка 1}\quad\quad\text{посылка 2}\quad\quad\dots}$\end{flushright}
\vspace{-0.7cm}
\item Аксиома:\\$\infer[\text{(акс.)}]{\Gamma,\alpha\vdash\alpha}{\vphantom{\Gamma}}$ 

\item Правила введения связок:\\$\infer{\Gamma\vdash\alpha\rightarrow\beta}{\Gamma,\alpha\vdash\beta}\quad\quad\infer{\Gamma\vdash\alpha\vee\beta}{\Gamma\vdash\alpha}$, $\infer{\Gamma\vdash\alpha\vee\beta}{\Gamma\vdash\beta}\quad\quad\infer{\Gamma\vdash\alpha\with\beta}{\Gamma\vdash\alpha\quad\quad\Gamma\vdash\beta}$

\item Правила удаления связок:\\$\infer{\Gamma\vdash\beta}{\Gamma\vdash\alpha\quad\Gamma\vdash\alpha\rightarrow\beta}\quad\quad\infer{\Gamma\vdash\gamma}{\Gamma\vdash\alpha\rightarrow\gamma\quad\Gamma\vdash\beta\rightarrow\gamma\quad\Gamma\vdash\alpha\vee\beta}$
 $\infer{\Gamma\vdash\alpha}{\Gamma\vdash\alpha\with\beta}\quad\quad\infer{\Gamma\vdash\beta}{\Gamma\vdash\alpha\with\beta}\quad\quad\infer{\Gamma\vdash\alpha}{\Gamma\vdash\bot}$
\item Пример доказательства:\vspace{-0.3cm}
$$\infer[(\text{введ}\with)]{A\with B\vdash B \with A}{\infer[(\text{удал}\with)]{A \with B \vdash B}{\infer[(\text{акс.})]{A \with B\vdash A \with B}{}}
                                           \quad\quad\infer[(\text{удал}\with)]{A \with B \vdash A}{\infer[(\text{акс.})]{A \with B\vdash A \with B}{}}}$$
\end{enumerate}

\subsection{Импликативный фрагмент интуиционистского исчисления высказываний.}
\noindent {\bf Определение.}  Импликационный фрагмент интуиционистской логики:

$$\infer{\Gamma,\varphi \vdash_\rightarrow \varphi}{} \quad\quad 
  \infer{\Gamma\vdash_\rightarrow\varphi\rightarrow\psi}{\Gamma,\varphi_\rightarrow\vdash\psi} \quad\quad 
  \infer{\Gamma\vdash_\rightarrow\psi}{\Gamma\vdash_\rightarrow\varphi\quad\quad\Gamma\vdash_\rightarrow\varphi\rightarrow\psi}$$


\subsection{Просто-типизированное лямбда исчисление.}
\noindent {\bf Теорема.} Если $\Gamma\vdash\alpha$, то $\Gamma\vdash_\rightarrow\alpha$.

\noindent Из корректности моделей Крипке следует, что что если $\Gamma\vdash\alpha$, то $\Gamma\Vdash \alpha$.
Требуемое следует из того, что $\Gamma\Vdash \alpha$ влечёт $\Gamma\vdash_\rightarrow\alpha$.

\subsection{Изоморфизм Карри-Ховарда (высказывание, доказательство, импликация, конъюнкция, дизъюнкция, ложь).}

\noindent {\bf Определение.}  Ложь ($\bot$) --- необитаемый тип;
$\texttt{failwith/raise/throw} : \alpha\rightarrow\bot$; $\neg\varphi\equiv\varphi\rightarrow\bot$


\noindent Например, контрапозиция:
$(\alpha\rightarrow\beta)\rightarrow(\neg\beta\rightarrow\neg\alpha)$

$$\infer[\lambda]{\lambda f^{\alpha\rightarrow\beta}.\lambda n^{\beta\rightarrow\bot}.\lambda a^\alpha.n\ (f\ a): (\alpha\rightarrow\beta)\rightarrow(\neg\beta\rightarrow\neg\alpha)}
        {\infer[\lambda]{f:\alpha\rightarrow\beta\vdash\lambda n^{\beta\rightarrow\bot}.\lambda a^\alpha.n\ (f\ a): \neg\beta\rightarrow\neg\alpha}
               {\infer[\lambda]{f:\alpha\rightarrow\beta,n:\beta\rightarrow\bot\vdash\lambda a^\alpha.n\ (f\ a): \neg\alpha}{
                       \infer[App]{f:\alpha\rightarrow\beta,n:\beta\rightarrow\bot, a:\alpha \vdash n\ (f\ a): \bot}{
                              \infer[App]{\Phi \vdash f\ a: \beta}{\infer[Ax]{\Phi \vdash a: \alpha}{}\quad\quad\infer[Ax]{\Phi \vdash f:\alpha\rightarrow\beta}{}}
    \quad\quad \infer[Ax]{\Phi \vdash n: \beta\rightarrow\bot}{}
               }
               }}}$$

Снятие двойного отрицания: $((\alpha\rightarrow\bot)\rightarrow\bot)\rightarrow\alpha$, то есть $\lambda f^{(\alpha\rightarrow\bot)\rightarrow\bot}.?: \alpha$.\\
$f$ угадывает, что передать $x: \alpha\rightarrow\bot$. Тогда надо по $f$ угадать, что передать $x$.

\end{document}