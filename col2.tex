\documentclass[10pt,a4paper,oneside]{article}
\usepackage[utf8]{inputenc}
\usepackage[english,russian]{babel}
\usepackage{amsmath}
\usepackage{amsthm}
\usepackage{amssymb}
\usepackage{enumerate}
\usepackage{stmaryrd}
\usepackage{cmll}
\usepackage{mathrsfs}
\usepackage[left=2cm,right=2cm,top=2cm,bottom=2cm,bindingoffset=0cm]{geometry}
\usepackage{proof}
\usepackage{tikz}
\usepackage{multicol}
\usepackage{mathabx}
\usepackage{comment}
\usepackage{hyperref}
\usepackage[utf8]{inputenc}
\usepackage[english,russian]{babel}
\usepackage{amssymb}
\usepackage{stmaryrd}
\usepackage{cmll}
\usepackage{xcolor}
\usepackage{proof}
\usetikzlibrary{hobby,fit,backgrounds,calc,shapes.geometric,patterns}
\usetikzlibrary{patterns}
\usepackage[normalem]{ulem}

\newcommand\doubleplus{+\kern-1.3ex+\kern0.8ex}
\newcommand\mdoubleplus{\ensuremath{\mathbin{+\mkern-10mu+}}}

\begin{document}

\section{Исчисление предикатов}
\subsection{Исчисление предикатов}
\subsubsection{Язык исчисления предикатов}
\begin{enumerate}
\item Два типа: предметные и логические выражения.
\item Предметные выражения: метапеременная {\color{blue}$\theta$}.
\begin{enumerate}
\item Предметные переменные: {\color{blue}$a$}, {\color{blue}$b$}, {\color{blue}$c$}, \dots, метапеременные {\color{blue}$x$}, {\color{blue}$y$}.
\item Функциональные выражения: {\color{blue}$f(\theta_1,\dots,\theta_n)$}, метапеременные {\color{blue}$f$}, {\color{blue}$g$}, \dots\\
%Имена разнообразны: {\color{blue}$(\theta_1+\theta_2)$}, {\color{blue}$0$} и т.п.
\item Примеры: %{\color{blue}$(x+1)^2$}. \\
% Раскроем сокращения: {\color{blue}$(\theta_0+\theta_1)\equiv p(\theta_0,\theta_1)$},
%   {\color{blue}$\theta_1^2\equiv q(\theta_1)$}, {\color{blue}$1\equiv r$}, {\color{blue}$2\equiv s$}. \\
  {\color{blue}$r$},
  {\color{blue}$q(p(x,s),r)$}.
\end{enumerate}
\item Логические выражения: метапеременные {\color{blue}$\alpha$}, {\color{blue}$\beta$}, {\color{blue}$\gamma$}, \dots
\begin{enumerate}
\item Предикатные выражения: {\color{blue}$P(\theta_1,\dots,\theta_n)$}, метапеременная {\color{blue}$P$}.\\
Имена: {\color{blue}$A$}, {\color{blue}$B$}, {\color{blue}$C$}, \dots  %также {\color{blue}$(\theta_1=\theta_2)$} и т.п.
\item Связки: {\color{blue}$(\varphi\vee\psi)$}, {\color{blue}$(\varphi\with\psi)$}, {\color{blue}$(\varphi\rightarrow\psi)$}, 
   {\color{blue}$(\neg\varphi)$}.
\item Кванторы: {\color{blue}$(\forall x.\varphi)$} и {\color{blue}$(\exists x.\varphi)$}.
\end{enumerate}
\end{enumerate}

\subsubsection{Сокращения записи, метаязык}

\begin{enumerate}
\item Метапеременные:
\begin{enumerate}
\item $\color{blue}\psi$, $\color{blue}\phi$, $\color{blue}\pi$, \dots --- формулы
\item $\color{blue}P$, $\color{blue}Q$, \dots --- предикатные символы
\item $\color{blue}\theta$, \dots --- термы
\item $\color{blue}f$, $\color{blue}g$, \dots --- функциональные символы
\item $\color{blue}x$, $\color{blue}y$, \dots --- предметные переменные
\end{enumerate}

\item Скобки --- как в И.В.; квантор --- жадный:
\begin{center}${\color{blue}(\forall a.} \underbrace{{\color{blue}A \vee B \vee C \rightarrow \exists b.}
                    \underbrace{\color{blue}D \with \neg E}_{\exists b.\dots}}_{\forall a.\dots} \color{blue}) \with F$\end{center}

\item Дополнительные обозначения при необходимости:
\begin{enumerate}
\item $\color{blue}(\theta_1 = \theta_2)$ вместо $\color{blue}E(\theta_1,\theta_2)$
\item $\color{blue}(\theta_1 + \theta_2)$ вместо $\color{blue}p(\theta_1,\theta_2)$
\item $\color{blue}0$ вместо $\color{blue}z$
\item \dots
\end{enumerate}

\end{enumerate}

\subsubsection{Оценка исчисления предикатов}
{\bf Определение.} Оценка --- упорядоченная четвёрка $\langle D, F, P, E \rangle$, где:

\begin{enumerate}
\item $D$ --- предметное множество;
\item $F$ --- оценка для функциональных символов; пусть $f_n$ --- $n$-местный функциональный символ:
 $$F_{f_n}: D^n \rightarrow D$$

\item $P$ --- оценка для предикатных символов; пусть $T_n$ --- $n$-местный предикатный символ:
 $$P_{T_n}: D^n \rightarrow V\quad\quad\quad V = \{\text{И}, \text{Л}\}$$

\item $E$ --- оценка для предметных переменных.
 $$E(x) \in D$$
\end{enumerate}

\subsubsection{Оценка формулы}

Запись и сокращения записи подобны исчислению высказываний: $$\llbracket \phi \rrbracket \in V,\quad
      \llbracket Q(x,f(x))\vee R\rrbracket^{x := 1, f(t) := t^2, R := \text{И}} = \text{И}$$

\begin{enumerate}
\item Правила для связок $\vee$, $\with$, $\neg$, $\rightarrow$ остаются прежние;
\item $\llbracket f_n (\theta_1, \theta_2, \dots, \theta_n) \rrbracket = F_{f_n} (\llbracket\theta_1\rrbracket,
          \llbracket\theta_2\rrbracket, \dots, \llbracket\theta_n\rrbracket)$
\item $\llbracket P_n (\theta_1, \theta_2, \dots, \theta_n) \rrbracket = P_{T_n} (\llbracket\theta_1\rrbracket,
          \llbracket\theta_2\rrbracket, \dots, \llbracket\theta_n\rrbracket)$
\item $$\llbracket \forall x.\phi \rrbracket = \left\{\begin{array}{ll}
   \text{И}, & \text{если } \llbracket\phi\rrbracket^{x := t} = \text{И}\text{ при всех } t \in D\\
   \text{Л}, & \text{если найдётся } t \in D, \text{ что } \llbracket\phi\rrbracket^{x := t} = \text{Л}
  \end{array}\right.$$
\item $$\llbracket \exists x.\phi \rrbracket = \left\{\begin{array}{ll}
   \text{И}, & \text{если найдётся } t \in D, \text{ что } \llbracket\phi\rrbracket^{x := t} = \text{И}\\
   \text{Л}, & \text{если } \llbracket\phi\rrbracket^{x := t} = \text{Л}\text{ при всех } t \in D
  \end{array}\right.$$
\end{enumerate}

\subsection{Общезначимость, следование, выводимость}
\subsubsection{Общезначимость}
{\bf Определение.} Формула исчисления предикатов общезначима, если истинна при любой оценке:
$$\models\phi$$
То есть истинна при любых $D$, $F$, $P$ и $E$.

\subsubsection{Следование, выводимость}
Рассмотрим язык исчисления предикатов. Возьмём все схемы аксиом классического исчисления высказываний и добавим ещё две схемы аксиом 
(здесь везде $\theta$ свободен для подстановки вместо $x$ в $\varphi$):

\begin{tabular}{ll}
11. & $(\forall x.\varphi) \rightarrow \varphi[x:=\theta]$\\
12. & $\varphi[x:=\theta] \rightarrow \exists x.\varphi$ 
\end{tabular}

Добавим ещё два правила вывода (здесь везде $x$ не входит свободно в $\varphi$):
$$\infer[\text{Правило для }\forall]{\varphi\rightarrow\forall x.\psi}{\varphi\rightarrow\psi}$$
$$\infer[\text{Правило для }\exists]{(\exists x.\psi)\rightarrow\varphi}{\psi\rightarrow\varphi}$$

{\bf Определение.} Доказуемость, выводимость, полнота, корректность --- аналогично исчислению высказываний.

\subsection{Теорема о дедукции для исчисления предикатов}
\subsubsection{Теорема}
{\bf Теорема.} Если $\Gamma\vdash\alpha\rightarrow\beta$, то $\Gamma,\alpha\vdash\beta$.
Если $\Gamma,\alpha\vdash\beta$ и в доказательстве не применяются правила для кванторов 
по свободным переменным из $\alpha$, то $\Gamma\vdash\alpha\rightarrow\beta$.

\subsubsection{Следование}
{\bf Определение.} $\gamma_1,\gamma_2,\dots,\gamma_n\models\alpha$, если выполнено два условия:
\begin{enumerate}
\item $\alpha$ выполнено всегда, когда выполнено $\gamma_1,\gamma_2,\dots,\gamma_n$;
\item $\alpha$ не использует кванторов по переменным, входящим свободно в $\gamma_1,\gamma_2,\dots,\gamma_n$.
\end{enumerate}

\noindent {\bf Теорема.} Если $\Gamma\vdash\alpha$ и в доказательстве не используются кванторы по свободным
переменным из $\Gamma$, то $\Gamma\models\alpha$

\subsection{Корректность}
\subsubsection{Теорема}
%\begin{thm}Если $\Gamma\vdash\alpha$,
%и $\alpha$ не использует кванторов по переменным, входящим
%свободно в $\Gamma$,
% то $\Gamma\models\alpha$\end{thm}
%\begin{thm}В любой модели для любой $\forall y.\varphi$ найдётся $\mathcal{y} \in \mathcal{D}$, что $\llbracket\forall y.\varphi\rrbracket = \llbracket\varphi\rrbracket^{y := \mathcal{y}}$\end{thm}
{\bf Теорема.} Если $\theta$ свободен для подстановки 
вместо $x$ в $\varphi$, то $\llbracket\varphi\rrbracket^{x := \llbracket\theta\rrbracket} = \llbracket\varphi[x := \theta]\rrbracket$

\section{Непротиворечивое множество формул}
\subsection{Непротиворечивое множество формул}
\subsubsection{Определение}
{\bf Определение.} $\Gamma$ --- \emph{непротиворечивое множество формул},
если $\Gamma\not\vdash\alpha\with\neg\alpha$ для любого $\alpha$

\subsubsection{Примеры}
\begin{enumerate}
\item непротиворечиво: \begin{itemize}
\item $\Gamma = \{A \rightarrow B \rightarrow A\}$
\item $\Gamma = \{P(x,y)\rightarrow\neg P(x,y), \forall x.\forall y.\neg P(x,y)\}$;
\end{itemize}
\item противоречиво: \begin{itemize}
\item $\Gamma = \{P\rightarrow\neg P, \neg P \rightarrow P\}$

так как
$P\rightarrow\neg P, \neg P \rightarrow P \ \vdash\  \neg P \with \neg\neg P$
\end{itemize}
\item пусть $D = \mathbb{Z}$ и $P(x) \equiv (x > 0)$, аналогом для этой модели
будет $\Gamma = \{P(1), P(2), P(3), \dots\}$
\end{enumerate}

\subsection{Полное непротиворечивое множество формул}
{\bf Определение.} $\Gamma$ --- \emph{полное} непротиворечивое множество замкнутых бескванторных формул,
если:
\begin{enumerate}\item $\Gamma$ содержит только замкнутые бескванторные формулы;
\item если $\alpha$ --- некоторая замкнутая бескванторная формула, то $\alpha\in\Gamma$ или $\neg\alpha\in\Gamma$.
\end{enumerate}

\noindent {\bf Определение.} $\Gamma$ --- \emph{полное} непротиворечивое множество замкнутых формул, если:
\begin{enumerate}\item $\Gamma$ содержит только замкнутые формулы;
\item если $\alpha$ --- некоторая замкнутая формула, то $\alpha \in \Gamma$, или $\neg\alpha \in \Gamma$.
\end{enumerate}

\noindent {\bf Теорема.} Пусть $\Gamma$ --- непротиворечивое множество замкнутых (бескванторных) формул. Тогда, какова бы ни была
замкнутая (бескванторная) формула $\varphi$, хотя бы $\Gamma \cup \{\varphi\}$ или $\Gamma \cup \{\neg\varphi\}$ ---
непротиворечиво.

\subsection{Доказательство существования моделей у непротиворечивых множеств формул 
в бескванторном исчислении предикатов.}
\subsubsection{Модель для множества формул}
\noindent {\bf Определение.} Моделью для множества формул $F$ назовём такую модель $\mathcal{M}$, что
    при всяком $\varphi \in F$ выполнено $\llbracket\varphi\rrbracket_\mathcal{M} = \text{И}$
    
\noindent Альтернативное обозначение: $\mathcal{M}\models\varphi$.

\subsubsection{Теорема}
\noindent {\bf Теорема.} Любое непротиворечивое множество замкнутых бескванторных формул имеет модель.    

\subsubsection{Доказательство теоремы о существовании модели}
{\bf Лемма.} Пусть $\varphi$ --- бескванторная формула, тогда $\mathcal{M}\models\varphi$ тогда и только тогда, когда $\varphi\in M$


\noindent {\bf Докозательство теоремы.}

Пусть $M$ --- непротиворечивое множество замкнутых бескванторных формул.

По теореме о пополнении существует $M'$ --- полное непротиворечивое множество замкнутых бескванторных формул,
что $M \subseteq M'$.

По лемме $M'$ имеет модель, эта модель подойдёт для $M$.

\subsection{Теорема Гёделя о полноте исчисления предикатов}
\subsubsection{Теорема}
{\bf Теорема.} Если $M$ --- замкнутое непротиворечивое множество формул, то оно имеет модель.

\subsection{Полнота исчисления предикатов}
\subsubsection{Следствие}
{\bf Следствие (из теоремы Гёделя о полноте).}
Исчисление предикатов полно.

\section{Машина Тьюринга. Задача об останове, её неразрешимость. Неразрешимость исчисления предикатов.}

\subsection{Машина Тьюринга.}
\noindent {\bf Определение.} Машина Тьюринга:
\begin{enumerate}
\item Внешний алфавит $q_1, \dots, q_n$, выделенный символ-заполнитель $q_\varepsilon$
\item Внутренний алфавит (состояний) $s_1, \dots, s_k$; $s_s$ --- начальное, $s_f$ --- допускающее, $s_r$ --- отвергающее.
\item Таблица переходов $\langle k, s \rangle \Rightarrow \langle k', s', \leftrightarrow \rangle$
\end{enumerate}

\noindent {\bf Определение.} Состояние машины Тьюринга:
\begin{enumerate}
\item Бесконечная лента с символом-заполнителем $q_\varepsilon$, текст конечной длины.
\item Головка над определённым символом.
\item Символ состояния (состояние в узком смысле) --- символ внутреннего алфавита.
\end{enumerate}

\subsection{Задача об останове, её неразрешимость.}
\subsubsection{Разрешимость.}
\noindent {\bf Определение.} Язык --- множество строк

\noindent {\bf Определение.} Язык $L$ разрешим, если существует машина Тьюринга, которая для любого слова $w$ переходит в допускающее состояние, если $w \in L$,
и в отвергающее, если $w \notin L$.

\subsubsection{Неразрешимость задачи останова.}
\noindent {\bf Определение.} Рассмотрим все возможные описания машин Тьюринга. Составим упорядоченные пары: описание машины Тьюринга и входная строка.
Из них выделим язык останавливающихся на данном входе машин Тьюринга.

\noindent{\bf Теорема.} Язык всех останавливающихся машин Тьюринга неразрешим.

\subsection{Неразрешимость исчисления предикатов: доказательство}
\noindent{\bf Теорема.} Язык всех доказуемых формул исчисления предикатов неразрешим
Т.е. нет машины Тьюринга, которая бы по любой формуле $\alpha$ определяла, доказуема ли она.

\noindent{\bf Доказательство.}  Пусть существует машина Тьюринга, разрешающая любую формулу.
На её основе тогда несложно построить некоторую машину Тьюринга, перестраивающую любую машину $S$ (с допускающим состоянием $s_f$ и входом $y$) 
в её ограничения $C$ и разрешающую формулу ИП $C \rightarrow \exists w_l.\exists w_r.F_{S,y}(w_l,w_r,s_f)$. 
Эта машина разрешит задачу останова.

\section{Порядок теории (0, 1, 2). Теории первого порядка. Аксиоматика Пеано. Арифметические операции. Формальная арифметика.}
\subsection{Порядок теории (0, 1, 2).}

\subsection{Аксиоматика Пеано.}
\subsubsection{Натуральные числа: аксиоматика Пеано, 1889}\vspace{-0.5cm}
 $$\mathbb{N}: 1, 2, \dots \mbox{ или } \mathbb{N}_0: 0, 1, 2, \dots$$\vspace{-1cm}
\noindent {\bf Определение.} 
  $N$ (или, более точно, $\langle N, 0, (')\rangle$) \emph{соответствует} аксиоматике Пеано, 
  если следующее определено/выполнено:
  \begin{enumerate}
     \item Операция <<штрих>> $('): N \to N$, причём нет $a,b \in N$, что $a \ne b$, но $a' = b'$.
           
           Если $x = y'$, то $x$ назовём следующим за $y$, а $y$ --- предшествующим $x$.
     \item Константа $0 \in N$: нет $x \in N$, что $x' = 0$.
     \item Индукция. Каково бы ни было свойство (<<предикат>>) $P: N \to V$, если:
           \begin{enumerate}
           \item $P(0)$
           \item При любом $x\in N$ из $P(x)$ следует $P(x')$
           \end{enumerate}
           то при любом $x \in N$ выполнено $P(x)$.
  \end{enumerate}
Как построить? Например, в стиле алгебры Линденбаума:
\begin{enumerate}
\item $N$ --- язык, порождённый грамматикой $\nu ::= \texttt{0}\ |\ \nu \texttt{<<'>>}$
\item $0$ --- это $\text{<<0>>}$, $x'$ --- это $x \doubleplus \texttt{<<'>>}$
\end{enumerate}

\subsection{Арифметические операции.}

\subsubsection{Обозначения и определения}
\noindent {\bf Определение.}
$1 = 0'$, $2 = 0''$, $3 = 0'''$, $4 = 0''''$, $5 = 0'''''$, $6 = 0''''''$,
$7 = 0'''''''$, $8 = 0''''''''$, $9 = 0'''''''''$

\noindent {\bf Определение.}\vspace{-0.3cm}
$$a + b = \left\{ \begin{array}{ll} a, & \mbox{если } b = 0\\
                                    (a + c)', & \mbox{если } b = c'
                  \end{array}\right.$$
\vspace{-0.3cm}
Например, $$2 + 2 = 0'' + 0'' = (0'' + 0')' = ((0'' + 0)')' = ((0'')')' = 0'''' = 4$$\vspace{-0.3cm}

\noindent {\bf Определение.}\vspace{-0.3cm}
$$a \cdot b = \left\{ \begin{array}{ll} 0, & \mbox{если } b = 0\\
                                    a \cdot c + a, & \mbox{если } b = c'
                  \end{array}\right.$$
                  
\subsubsection{Коммутативность сложения.}
\noindent{\bf Теорема.} 
$a + b = b + a$

\subsection{Формальная арифметика.}
\noindent {\bf Определение.} 
Формальная арифметика --- теория первого порядка, со следующими добавленными нелогическими \dots
\begin{enumerate}
\item двухместными функциональными символами $(+)$, $(\cdot)$; одноместным функциональным символом $(')$, 
нульместным функциональным символом $0$;
\item двухместным предикатным символом $(=)$;
\item восемью нелогическими \emph{аксиомами}:\vspace{0.1cm}
\begin{tabular}{ll}
(A1) $a=b \to a=c \to b=c$             &(A5) $a+0 = a$                     \\
(A2) $a=b \to a'=b'$                   &(A6) $a+b' = (a+b)'$               \\
(A3) $a'=b' \to a=b$                   &(A7) $a\cdot 0 = 0$                \\
(A4) $\neg a' = 0$                     &(A8) $a\cdot b' = a \cdot b + a$
\end{tabular}
\item нелогической схемой аксиом индукции $\psi[x:=0]\with(\forall x.\psi\to \psi[x:=x'])\to \psi$ с метапеременными $x$ и $\psi$.
\end{enumerate}

\end{document}