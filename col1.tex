\documentclass[10pt,a4paper,oneside]{article}
\usepackage[utf8]{inputenc}
\usepackage[english,russian]{babel}
\usepackage{amsmath}
\usepackage{amsthm}
\usepackage{amssymb}
\usepackage{enumerate}
\usepackage{stmaryrd}
\usepackage{cmll}
\usepackage{mathrsfs}
\usepackage[left=2cm,right=2cm,top=2cm,bottom=2cm,bindingoffset=0cm]{geometry}
\usepackage{proof}
\usepackage{tikz}
\usepackage{multicol}
\usepackage{mathabx}
\usepackage{comment}
\usepackage{hyperref}
\usepackage[utf8]{inputenc}
\usepackage[english,russian]{babel}
\usepackage{amssymb}
\usepackage{stmaryrd}
\usepackage{cmll}
\usepackage{xcolor}
\usepackage{proof}
\usepackage[normalem]{ulem}

\begin{document}

\section{Исчисление высказываний}

\subsection{Предметный язык и язык исследователя (метаязык). Соглашения об обозначениях. Схемы формул.}
Высказывание --- это строка, сформированная по следующим правилам.

\subsection{Оценка высказываний, общезначимость, следование.}

\subsubsection{}
Чтобы задать оценку высказываний:
Зафиксируем множество истинностных значений $V = \{\textit{И},\textit{Л}\,\}$

Определим функцию оценки переменных (\emph{интерпретацию}) $f: P \rightarrow V$\\
(P --- множество пропозициональных переменных).
Если $\llbracket A \rrbracket = \textit{Л}$ и $\llbracket B \rrbracket = \textit{И}$,
то $\llbracket (A\rightarrow B)\rightarrow (B\rightarrow A) \rrbracket = \textit{Л}$

\subsubsection{}
Синтаксис для указания функции оценки переменных
$$\llbracket \alpha \rrbracket^{X_1 := v_1,\ \dots,\ X_n := v_n}$$
Это всё метаязык --- потому полагаемся на здравый смысл
$$\llbracket A \with B \with (C \rightarrow C) \rrbracket^{A := \textit{И},\ B := \llbracket \neg A \rrbracket}$$

\subsubsection{}

\begin{enumerate}
    \item Переменные $$\llbracket X \rrbracket = f(X)\quad\quad\quad \llbracket X \rrbracket^{X := a} = a$$ \vspace{-0.3cm}
    \item Отрицание $$\llbracket \neg \alpha \rrbracket = 
      \left\{\begin{array}{ll}\textit{Л},&\textit{если }\llbracket\alpha\rrbracket=\textit{И}\\
                            \textit{И},&\textit{иначе}\end{array}\right.$$ \vspace{-0.1cm}
    \item Конъюнкция $$\llbracket \alpha \with \beta \rrbracket = 
      \left\{\begin{array}{ll}\textit{И},&\textit{если }\llbracket\alpha\rrbracket=\llbracket\beta\rrbracket=\textit{И}\\ 
                            \textit{Л},&\textit{иначе}\end{array}\right.$$ \vspace{-0.1cm}
    \item Дизъюнкция $$\llbracket \alpha \vee \beta \rrbracket = 
      \left\{\begin{array}{ll}\textit{Л},&\textit{если }\llbracket\alpha\rrbracket=\llbracket\beta\rrbracket=\textit{Л}\\
                            \textit{И},&\textit{иначе}\end{array}\right.$$ \vspace{-0.1cm}
    \item Импликация $$\llbracket \alpha \rightarrow \beta \rrbracket = 
      \left\{\begin{array}{ll}\textit{Л},&\textit{если }\llbracket\alpha\rrbracket=\textit{И},\ \llbracket\beta\rrbracket=\textit{Л}\\
                            \textit{И},&\textit{иначе}\end{array}\right.$$
\end{enumerate}

\subsubsection{}

Если $\alpha$ истинна при любой оценке переменных, то она \emph{общезначима} (является \emph{тавтологией}):
$$\models \alpha$$


Выражение $A\rightarrow A$ --- тавтология. 
Переберём все возможные значения единственной переменной $A$:

$$
\begin{array}{l} \llbracket A\rightarrow A \rrbracket ^ {A := \textit{И}} = \textit{И} \\
 \llbracket A\rightarrow A \rrbracket ^ {A := \textit{Л}} = \textit{И} \end{array}
$$

Выражение $A\rightarrow\neg A$ тавтологией не является:

$$\llbracket A\rightarrow\neg A \rrbracket ^ {A := \textit{И}} = \textit{Л}$$

\subsubsection{}
\begin{enumerate}
    \item Если $\alpha$ истинна при любой оценке переменных, при которой истинны 
    высказывания $\gamma_1, \dots, \gamma_n$, будем говорить, что $\alpha$ --- \emph{следствие} этих высказываний:
    $$\gamma_1, \dots, \gamma_n \models \alpha$$
    \item Истинна при какой-нибудь оценке --- \emph{выполнима}.
    \item Не истинна ни при какой оценке --- \emph{невыполнима}.
    \item Не истинна при какой-нибудь оценке --- \emph{опровержима}.
\end{enumerate}

\subsection{Доказуемость, гипотезы (контекст), выводимость.}
\subsubsection{}
Доказательством (выводом) назовём конечную последовательность высказываний $\delta_1, \delta_2, \dots, \delta_n$,
причём каждое $\delta_i$ либо:
\begin{enumerate}
\item является аксиомой --- существует замена метапеременных для какой-либо схемы аксиом, позволяющая получить
формулу $\delta_i$, либо
\item получается из $\delta_1,\dots,\delta_{i-1}$ по правилу Modus Ponens --- существуют такие индексы $j < i$ и $k < i$,
что $\delta_k \equiv \delta_j\rightarrow\delta_i$.
\end{enumerate}

\subsubsection{}

(доказательство формулы $\alpha$)
--- такое доказательство (вывод) $\delta_1, \delta_2, \dots, \delta_n$,
что $\alpha\equiv\delta_n$.

Формула $\alpha$ доказуема (выводима), если существует её доказательство. Обозначение:
$$\vdash \alpha$$

\subsubsection{}
(вывод формулы $\alpha$ из гипотез $\gamma_1,\dots,\gamma_k$)
    --- такая последовательность
    $\delta_1,\dots,\delta_n$, причём каждое $\delta_i$ либо:
    \begin{enumerate}
    \item является аксиомой;
    \item либо получается по правилу Modus Ponens из предыдущих;
    \item либо является одной из гипотез: существует $t: \delta_i \equiv \gamma_t$.
    \end{enumerate}


\section{Корректность, полнота, противоречивость и непротиворечивость (эквивалентные формулировки).}


\end{document}
