\documentclass[10pt,a4paper,oneside]{article}
\usepackage[utf8]{inputenc}
\usepackage[english,russian]{babel}
\usepackage{amsmath}
\usepackage{amsthm}
\usepackage{amssymb}
\usepackage{enumerate}
\usepackage{stmaryrd}
\usepackage{cmll}
\usepackage{mathrsfs}
\usepackage[left=2cm,right=2cm,top=2cm,bottom=2cm,bindingoffset=0cm]{geometry}
\usepackage{proof}
\usepackage{tikz}
\usepackage{multicol}
\usepackage{mathabx}
\usepackage{comment}
\usepackage{hyperref}
\usepackage[utf8]{inputenc}
\usepackage[english,russian]{babel}
\usepackage{amssymb}
\usepackage{stmaryrd}
\usepackage{cmll}
\usepackage{xcolor}
\usepackage{proof}
\usetikzlibrary{hobby,fit,backgrounds,calc,shapes.geometric,patterns}
\usepackage[normalem]{ulem}

\begin{document}

\section{Исчисление высказываний}

\subsection{Предметный язык и язык исследователя (метаязык). Соглашения об обозначениях. Схемы формул.}


\subsection{Язык исчисления высказываний.}
\noindent \textbf{ Определение. }
Высказывание --- это строка, сформированная по следующим правилам.

\begin{enumerate}
\item Атомарное высказывание --- пропозициональная переменная: $A, B', C_{1234}$ 

\item Составное высказывание: если $\alpha$ и $\beta$ --- высказывания, то высказываниями являются:
\begin{enumerate}
\item Отрицание: $(\neg\alpha)$ 
\item Конъюнкция: $(\alpha\with\beta)\ \ \textit{или}\ \ (\alpha\wedge\beta)$ 
\item Дизъюнкция: $(\alpha\vee\beta)$ 
\item Импликация: $(\alpha\rightarrow\beta)\ \ \textit{или}\ \ (\alpha\supset\beta)$ 
\end{enumerate}
\end{enumerate}

\noindent \textbf{ Пример. }
$$(((A\rightarrow B)\vee (B\rightarrow C)) \vee (C \rightarrow A))$$

\subsection{Оценка высказываний, общезначимость, следование.}

\subsubsection{Оценка высказываний.}
\noindent \textbf{ Определение. }
Чтобы задать оценку высказываний:
Зафиксируем множество истинностных значений $V = \{\textit{И},\textit{Л}\,\}$

Определим функцию оценки переменных (\emph{интерпретацию}) $f: P \rightarrow V$\\
(P --- множество пропозициональных переменных).
Если $\llbracket A \rrbracket = \textit{Л}$ и $\llbracket B \rrbracket = \textit{И}$,
то $\llbracket (A\rightarrow B)\rightarrow (B\rightarrow A) \rrbracket = \textit{Л}$

\subsubsection{Синтаксис.}
\noindent \textbf{ Определение. }
Синтаксис для указания функции оценки переменных
$$\llbracket \alpha \rrbracket^{X_1 := v_1,\ \dots,\ X_n := v_n}$$
Это всё метаязык --- потому полагаемся на здравый смысл
$$\llbracket A \with B \with (C \rightarrow C) \rrbracket^{A := \textit{И},\ B := \llbracket \neg A \rrbracket}$$

\subsubsection{Оценка высказываний рекурсивно.}

\begin{enumerate}
    \item Переменные $$\llbracket X \rrbracket = f(X)\quad\quad\quad \llbracket X \rrbracket^{X := a} = a$$ \vspace{-0.3cm}
    \item Отрицание $$\llbracket \neg \alpha \rrbracket = 
      \left\{\begin{array}{ll}\textit{Л},&\textit{если }\llbracket\alpha\rrbracket=\textit{И}\\
                            \textit{И},&\textit{иначе}\end{array}\right.$$ \vspace{-0.1cm}
    \item Конъюнкция $$\llbracket \alpha \with \beta \rrbracket = 
      \left\{\begin{array}{ll}\textit{И},&\textit{если }\llbracket\alpha\rrbracket=\llbracket\beta\rrbracket=\textit{И}\\ 
                            \textit{Л},&\textit{иначе}\end{array}\right.$$ \vspace{-0.1cm}
    \item Дизъюнкция $$\llbracket \alpha \vee \beta \rrbracket = 
      \left\{\begin{array}{ll}\textit{Л},&\textit{если }\llbracket\alpha\rrbracket=\llbracket\beta\rrbracket=\textit{Л}\\
                            \textit{И},&\textit{иначе}\end{array}\right.$$ \vspace{-0.1cm}
    \item Импликация $$\llbracket \alpha \rightarrow \beta \rrbracket = 
      \left\{\begin{array}{ll}\textit{Л},&\textit{если }\llbracket\alpha\rrbracket=\textit{И},\ \llbracket\beta\rrbracket=\textit{Л}\\
                            \textit{И},&\textit{иначе}\end{array}\right.$$
\end{enumerate}

\subsubsection{Тавтологии}

Если $\alpha$ истинна при любой оценке переменных, то она \emph{общезначима} (является \emph{тавтологией}):
$$\models \alpha$$


Выражение $A\rightarrow A$ --- тавтология. 
Переберём все возможные значения единственной переменной $A$:

$$
\begin{array}{l} \llbracket A\rightarrow A \rrbracket ^ {A := \textit{И}} = \textit{И} \\
 \llbracket A\rightarrow A \rrbracket ^ {A := \textit{Л}} = \textit{И} \end{array}
$$

Выражение $A\rightarrow\neg A$ тавтологией не является:

$$\llbracket A\rightarrow\neg A \rrbracket ^ {A := \textit{И}} = \textit{Л}$$

\subsubsection{Определения}
\begin{enumerate}
    \item Если $\alpha$ истинна при любой оценке переменных, при которой истинны 
    высказывания $\gamma_1, \dots, \gamma_n$, будем говорить, что $\alpha$ --- \emph{следствие} этих высказываний:
    $$\gamma_1, \dots, \gamma_n \models \alpha$$
    \item Истинна при какой-нибудь оценке --- \emph{выполнима}.
    \item Не истинна ни при какой оценке --- \emph{невыполнима}.
    \item Не истинна при какой-нибудь оценке --- \emph{опровержима}.
\end{enumerate}

\subsection{Доказуемость, гипотезы (контекст), выводимость.}
\subsubsection{Вывод}
\noindent \textbf{ Определение. }
Доказательством (выводом) назовём конечную последовательность высказываний $\delta_1, \delta_2, \dots, \delta_n$,
причём каждое $\delta_i$ либо:
\begin{enumerate}
\item является аксиомой --- существует замена метапеременных для какой-либо схемы аксиом, позволяющая получить
формулу $\delta_i$, либо
\item получается из $\delta_1,\dots,\delta_{i-1}$ по правилу Modus Ponens --- существуют такие индексы $j < i$ и $k < i$,
что $\delta_k \equiv \delta_j\rightarrow\delta_i$.
\end{enumerate}

\subsubsection{Доказуемость}
\noindent \textbf{ Определение. }
(доказательство формулы $\alpha$)
--- такое доказательство (вывод) $\delta_1, \delta_2, \dots, \delta_n$,
что $\alpha\equiv\delta_n$.

Формула $\alpha$ доказуема (выводима), если существует её доказательство. Обозначение:
$$\vdash \alpha$$

\subsubsection{Выводимость}
\noindent \textbf{ Определение. }
(вывод формулы $\alpha$ из гипотез $\gamma_1,\dots,\gamma_k$)
    --- такая последовательность
    $\delta_1,\dots,\delta_n$, причём каждое $\delta_i$ либо:
    \begin{enumerate}
    \item является аксиомой;
    \item либо получается по правилу Modus Ponens из предыдущих;
    \item либо является одной из гипотез: существует $t: \delta_i \equiv \gamma_t$.
    \end{enumerate}

\subsubsection{Контекст}
\noindent \textbf{ Определение. }
Будем обозначать большими греческими буквами середины
алфавита, возможно с индексами, ($\Gamma$, $\Delta_1$, ...) списки формул.
Будем использовать, где удобно:

$$\Gamma \vdash \alpha$$

Списки можно указывать через запятую:

$$\Gamma, \Delta, \zeta \vdash \alpha$$
это означает то же, что и 
$$\gamma_1,\gamma_2,\dots,\gamma_n,\delta_1,\delta_2,\dots,\delta_m,\zeta\vdash\alpha$$
если 
$$\Gamma := \{\gamma_1,\gamma_2,\dots,\gamma_n\},\quad \Delta := \{\delta_1,\delta_2,\dots,\delta_m\}$$


\subsection{Корректность, полнота, противоречивость и непротиворечивость (эквивалентные формулировки).}

\subsubsection{Корректность}
\textbf{ Лемма. }
Теория корректна, если любое доказуемое в ней утверждение общезначимо.
То есть, $\vdash\alpha$ влечёт $\models\alpha$.


\noindent \textbf{ Лемма. } Если $\vdash\alpha$, то $\models\alpha$

\subsubsection{Полнота}
\noindent \textbf{ Лемма. } Теория полна, если любое общезначимое в ней утверждение доказуемо.
То есть, $\models\alpha$ влечёт $\vdash\alpha$.

\subsection{Теорема о дедукции для исчисления высказываний (формулировка). Теорема о полноте исчисления высказываний (формулировка)}

\subsubsection{Теорема о дедукции для исчисления высказываний}

\noindent \textbf{ Теорема. }
$\Gamma,\alpha\vdash\beta$ выполнено тогда и только тогда, когда выполнено $\Gamma\vdash\alpha\rightarrow\beta$

\subsubsection{Теорема о полноте исчисления высказываний}
\noindent \textbf{ Теорема. }
Если $\models\alpha$, то $\vdash\alpha$.

\section{Топологическое пространство}
\subsection{Определение.}
\subsubsection{}
\noindent \textbf{ Определение. }
Топологическим пространством называется упорядоченная пара $\langle X, \Omega \rangle$,
где $X$ --- некоторое множество, а $\Omega \subseteq \mathcal{P}(X)$, причём:
\begin{enumerate}
\item $\varnothing, X \in \Omega$
\item если $A_1, \dots, A_n \in \Omega$, то $A_1 \cap A_2 \cap \dots \cap A_n \in \Omega$;
\item если $\{A_\alpha\}$ --- семейство множеств из $\Omega$, то и $\bigcup_\alpha A_\alpha \in \Omega$.
\end{enumerate}

Множество $\Omega$ называется \emph{топологией}.
Элементы $\Omega$ называются открытыми множествами.

\subsection{Метрическое пространство.}
\subsubsection{}
\noindent \textbf{ Определение. }
Метрикой на $X$ назовём множество, на котором определена функция расстояния $d: X^2 \rightarrow \mathbb{R}^+$, 
удовлетворяющая следующим свойствам:
\begin{enumerate}
\item $d(x,y) = 0$ тогда и только тогда, когда $x = y$
\item $d(x,y) = d(y,x)$
\item $d(x,z) \le d(x,y) + d(y,z)$ (неравенство треугольника)
\end{enumerate}

\noindent \textbf{ Определение. }
Открытым $\varepsilon$-шаром с центром в точке $x \in X$ назовём $O_\varepsilon(x) = \{ t \in X\ |\ d(x,t) < \varepsilon \}$.


\noindent \textbf{ Определение. }
Если $X$ --- некоторое множество и $d$ --- метрика на $X$, то будем говорить, что топологическое
пространство, задаваемое базой $\mathcal{B} = \{ O_\varepsilon(x)\ |\ \varepsilon \in \mathbb{R}^+, x \in X \}$,
порождено метрикой $d$.

\subsection{Примеры (топология стрелки, Зарисского, топология на дереьвях). }
\subsubsection{Топология стрелки}


\noindent \textbf{ Определение. } Топология стрелки: $\langle \mathbb{R}, \{(x,+\infty)\ |\ x\in\mathbb{R}\}\cup\{\varnothing,\mathbb{R}\}\rangle$ --- открыты все положительные лучи.


\subsubsection{Топология Зарисского}

\subsubsection{Топология на дереьвях}

\noindent \textbf{ Определение. } Пусть некоторый лес задан конечным множеством вершин $V$ и
отношением $(\preceq)$, связывающим предков и потомков ($a \preceq b$, если $b$ --- потомок $a$). Тогда подмножество его вершин $X\subseteq V$ назовём открытым, 
если из $a \in X$ и $a \preceq b$ следует, что $b \in X$.

\noindent \textbf{ Лемма. } Лес связен (является одним деревом) тогда и только тогда, когда соответствующее ему 
топологическое пространство связно.

\subsection{Открытые и замкнутые множества. Связность. Компактность. }

\subsubsection{Открытые и замкнутые множества.}

\noindent \textbf{ Определение. } Множество $\Omega$ называется \emph{топологией}.
\noindent \textbf{ Определение. } Элементы $\Omega$ называются открытыми множествами.

\subsubsection{Связность.}

\noindent \textbf{ Определение. } Пространство $\langle X, \Omega\rangle$ связно, если нет $A,B \in \Omega$, что $A\cup B = X$,
$A \cap B = \varnothing$ и $A,B \ne \varnothing$.

\subsubsection{Компактность.}

\noindent \textbf{ Определение. } Будем говорить, что множество компактно, если из любого его открытого покрытия можно выбрать конечное
подпокрытие.

\noindent \textbf{ Пример. } Множество $\{0,1\}$ в дискретной топологии компактно.


\noindent \textbf{ Пример. } Интервал $(0,1)$ в $\mathbb{R}$ не компактен --- например, рассмотрим покрытие $\{(\varepsilon,1)\ |\ \varepsilon\in(0,1)\}$ 

\subsection{Непрерывные функции. Путь. Линейная связность. }
\subsubsection{Непрерывные функции.}

\noindent \textbf{ Определение. } 
Функция $f: X \rightarrow Y$ непрерывна, если прообраз любого открытого множества открыт.

\noindent \textbf{ Пример. } 
Функция $f: \mathbb{N} \rightarrow \mathbb{R}$ всегда непрерывна (при дискретной топологии на $\mathbb{N}$), 
поскольку любое множество в $\mathbb{N}$ открыто.

\subsubsection{Путь.}

\subsubsection{Линейная связность.}


\section{Интуиционистское исчисление высказываний}
\subsection{Доказательства чистого существования.}

\noindent \textbf{ Теорема. }
Любое непрерывное отображение $f$ шара в $\mathbb{R}^n$ на себя имеет неподвижную точку

\noindent \textbf{ Теорема. }
Существует пара иррациональных чисел $a$ и $b$,
такая, что $a^b$ — рационально.

\subsection{BHK-интерпретация.}
BHK — это сокращение трёх фамилий: Брауэр, Гейтинг, Колмогоров.

\begin{enumerate}

\item $\alpha\ \&\ \beta$ построено, если построены $\alpha$ и $\beta$
\item $\alpha \vee \beta$ построено, если построено $\alpha$ или $\beta$,
и мы знаем, что именно
\item $\alpha\rightarrow\beta$ построено, если есть способ перестроения
$\alpha$ в $\beta$
\item $\bot$ — конструкция, не имеющая построения
\item $\neg\alpha$ построено, если построено $\alpha\rightarrow\bot$

\end{enumerate}


\subsection{Решётки.}
\noindent \textbf{ Определение. } 
Решёткой называется упорядоченная пара: $\langle X, (\preceq)\rangle$, 
где $X$ --- некоторое множество, а $(\preceq)$ --- частичный порядок на $X$, такой, 
что для любых $a,b \in X$ определены $a + b = \sup\{a,b\}$ и $a \cdot b = \inf\{a,b\}$.

\noindent \textbf{ Пример. } 
$\langle\Omega, (\subseteq)\rangle$ --- решётка.
$\langle\mathbb{N}\setminus\{1\}, (\raisebox{-0.75pt}{\vdots})\rangle$ --- не решётка.

\subsection{Дистрибутивная решётка. Пентагон и диамант.}

\noindent \textbf{ Определение. } 
Дистрибутивной решёткой называется такая, что для любых $a,b,c$ выполнено
$a \cdot (b + c) = a \cdot b + a \cdot c$.

\noindent \textbf{ Определение. } 
Импликативная решётка --- такая, в которой для любых элементов есть псевдодополнение.

\noindent \textbf{ Лемма. } 
Любая импликативная решётка --- дистрибутивна.

\noindent \textbf{ Определение. }
Псевдодополнением $a \rightarrow b$ называется наибольший из $\{ x \ |\ a \cdot x \preceq b\}$.

\noindent \textbf{ Пример. } 

\begin{center}\tikz{
  \node[circle,inner sep=0.3] (A) at (1,-1.3) {$a$};
  \node[fill=cyan!0.2,circle,inner sep=0.3] (B) at (0.5,-0.5) {$b$};
  \node[circle,inner sep=0.3] (C) at (1.5,-0.5) {$c$};
  \node[circle,inner sep=0.3] (D) at (1,0.3) {$d$};
  \foreach \b/\e in {A/B, A/C, B/D, C/D} {
      \draw[stealth-, line width=1, color=black!50!green] (\b) to (\e);
  }

  \node (txt) at (4,-0.5) { \begin{tabular}{l} $a \cdot b = a$ \\ $b \cdot b = b$ \\ $c \cdot b = a$ \\ $d \cdot b = b$ \end{tabular} };
  \draw[fill=cyan,fill=cyan,opacity=0.2](C.north west) 
     to[closed,curve through={
       (A.south west) .. (A.south east)
     }] (C.north east);
}\end{center}

Здесь $b \rightarrow c = \text{наиб}\{x\ |\ b \cdot x \preceq c\} = \text{наиб}\{ a, c \} = c$


\noindent \textbf{ Пример. (нет псевдодополнения: диамант и пентагон)}

\begin{center}\tikz{
  \node[circle,inner sep=0.3] (A) at (1,-1.3) {$a$};
  \node[circle,inner sep=0.3] (B) at (0,-0.5) {$b$};
  \node[circle,inner sep=0.3] (C) at (1,-0.5) {$c$};
  \node[circle,inner sep=0.3] (D) at (2,-0.5) {$d$};
  \node[circle,inner sep=0.3] (E) at (1,0.3) {$e$};
  \foreach \b/\e in {A/B, A/C, A/D, B/E, C/E, D/E} {
      \draw[stealth-, line width=1, color=black!50!green] (\b) to (\e);
  }
  \draw[pattern=north west lines,opacity=0.5,pattern color=olive](C.north west) 
     to[closed,curve through={
       (A.south west) .. (A.south east) .. (D.south east)
     }] (D.north east);
  \node (BC) at (1,-1.8) {$b \rightarrow c = \text{наиб}\{ a, c, d \}$};

  \node[circle,inner sep=0.3] (A1) at (9,-1.3) {$a$};
  \node[circle,inner sep=0.3] (B1) at (8,-1) {$b$};
  \node[circle,inner sep=0.3] (C1) at (8,0) {$c$};
  \node[circle,inner sep=0.3] (D1) at (10,-0.5) {$d$};
  \node[circle,inner sep=0.3] (E1) at (9,0.3) {$e$};
  \foreach \b/\e in {A1/B1, B1/C1, C1/E1, A1/D1, D1/E1} {
      \draw[stealth-, line width=1, color=black!50!green] (\b) to (\e);
  }

  \node (B1C1) at (9,-1.8) {$c \rightarrow b = \text{наиб}\{ a, b, d \}$};
  \draw[pattern=north west lines,opacity=0.5,pattern color=olive](B1.north west) 
     to[closed,curve through={
       (A1.south east) .. (D1.south east) .. (D1.north east)
     }] (B1.north east);
}\end{center}


\subsection{Булевы и псевдобулевы алгебры.}
\noindent \textbf{ Определение. }
0 --- наименьший элемент решётки, а 1 --- наибольший элемент решётки.

\noindent \textbf{ Лемма. }
В любой импликативной решётке $\langle X, (\preceq)\rangle$ есть 1.

\noindent \textbf{ Определение. }
Импликативная решётка с 0 --- псевдобулева алгебра (алгебра Гейтинга).
В такой решётке определено $\sim a := a \rightarrow 0$.

\noindent \textbf{ Определение. }
Булева алгебра --- псевдобулева алгебра, в которой $a\ + \sim a = 1$ для всех $a$.

\subsection{Алгебра Линденбаума.}
\noindent \textbf{ Определение. }
Определим предпорядок на высказываниях: $\alpha \preceq \beta := \alpha \vdash \beta$ в интуиционистском исчислении высказываний.
Также $\alpha\approx\beta$, если $\alpha\preceq\beta$ и $\beta\preceq\alpha$.

\noindent \textbf{ Определение. }
Пусть $L$ --- множество всех высказываний. Тогда алгебра Линденбаума $\mathcal{L} = L/_\approx$.

\noindent \textbf{ Теорема. }
$\mathcal{L}$ --- псевдобулева алгебра.

\subsection{Полнота интуиционистского исчисления высказываний в псевдобулевых алгебрах (формулировка, идея доказательства).}
\noindent \textbf{ Теорема. }
Пусть $\llbracket\alpha\rrbracket = [\alpha]_\mathcal{L}$.
Такая оценка интуиционистского исчисления высказываний алгеброй Линденбаума является согласованной.

\noindent \textbf{ Теорема. }
Интуиционистское исчисление высказываний полно в псевдобулевых алгебрах:
если $\models\alpha$ во всех псевдобулевых алгебрах, то $\vdash\alpha$. 

\subsection{Модели Крипке. Вынужденность.}
\noindent \textbf{ Определение. }
Модель Крипке $\langle \mathcal{W}, \preceq, (\Vdash)\rangle$:
\begin{enumerate}
\item $\mathcal{W}$ --- множество миров, $(\preceq)$ --- нестрогий частичный порядок на $\mathcal{W}$;
\item $(\Vdash)\subseteq \mathcal{W}\times P$ --- отношение вынуждения
между мирами и переменными, причём, если $W_i \preceq W_j$ и $W_i \Vdash X$, то $W_j \Vdash X$.
\end{enumerate}


\noindent Доопределим вынужденность:
\begin{enumerate}
\item $W \Vdash \alpha\with\beta$, если $W \Vdash \alpha$ и $W \Vdash \beta$;
\item $W \Vdash \alpha\vee\beta$, если $W \Vdash \alpha$ или $W \Vdash \beta$;
\item $W \Vdash \alpha\rightarrow\beta$, если всегда при $W \preceq W_1$ и $W_1 \Vdash \alpha$ выполнено $W_1 \Vdash \beta$
\item $W \Vdash \neg\alpha$, если всегда при $W \preceq W_1$ выполнено $W_1 \not\Vdash \alpha$.
\end{enumerate}

Будем говорить, что $\Vdash\alpha$, если $W\Vdash\alpha$ при всех $W \in \mathcal{W}$.
Будем говорить, что $\models_\kappa\alpha$, если $\Vdash\alpha$ во всех моделях Крипке.

\subsection{Сведение моделей Крипке к псевдобулевым алгебрам.}
\noindent \textbf{ Лемма. }
Если $W_1 \Vdash \alpha$ и $W_1 \preceq W_2$, то $W_2 \Vdash \alpha$

\noindent \textbf{ Теорема. }
Пусть $\langle \mathcal{W}, (\preceq), (\Vdash)\rangle$ ---
некоторая модель Крипке.
Тогда она есть корректная модель интуиционистского исчисления высказываний.


\subsection{Нетабличность ИИВ (формулировка теоремы).}

\noindent \textbf{ Определение. }
Пусть задано $V$, значение $T \in V$ (<<истина>>), функция $f_P: P \rightarrow V$, 
функции $f_\with, f_\vee, f_\rightarrow : V \times V \rightarrow V$,
функция $f_\neg: V \rightarrow V$.

Тогда оценка $\llbracket X \rrbracket = f_P(X)$, 
$\llbracket \alpha\star\beta \rrbracket = f_\star(\llbracket \alpha \rrbracket, \llbracket \beta \rrbracket)$,
$\llbracket \neg\alpha \rrbracket = f_\neg(\llbracket\alpha\rrbracket)$ --- табличная.

\noindent \textbf{ Определение. }
Табличная модель конечна, если $V$ конечно.

\noindent \textbf{ Теорема. }
Не существует полной конечной табличной модели для интуиционистского исчисления высказываний.

\section{Дизъюнктивность интуиционистского исчисления высказываний.}
\subsection{Гёделева алгебра. Операция $\Gamma(A)$.}
\noindent \textbf{ Определение. }
Для алгебры Гейтинга $\mathcal{A} = \langle A, (\preceq) \rangle$ определим операцию <<гёделевизации>>: 
$\Gamma(\mathcal{A}) = \langle A\cup\{\omega\}, (\preceq_{\Gamma(\mathcal{A})}) \rangle$, где
отношение $(\preceq_{\Gamma(\mathcal{A})})$ --- минимальное отношение порядка,
удовлетворяющее условиям:

\vspace{-0.5cm}
\begin{center}\begin{tabular}{cc}
\begin{minipage}{9cm}
\begin{enumerate}
\item $a \preceq_{\Gamma(\mathcal{A})} b$, если $a \preceq_\mathcal{A} b$ и $a,b \notin \{\omega,1\}$;
\item $a \preceq_{\Gamma(\mathcal{A})} \omega$, если $a \ne 1$;
\item $\omega \preceq_{\Gamma(\mathcal{A})} 1$
\end{enumerate}
\end{minipage}
&
\begin{minipage}{4cm}\begin{center}
\tikz{
    \filldraw[pattern=north west lines,pattern color=gray] (1,-1) circle (1cm);
    \node[right] at (2.2,-1) (A) {$A \setminus \{1\}$};
    \node[circle,fill,inner sep=2pt, outer sep=0pt,label=right:$1$] at (1,1) (Max) {};
    \node[circle,fill,inner sep=2pt, outer sep=0pt,label=above right:$\omega$] at (1,0) (Omega) {}; 
    \draw[-stealth,line width=1] (Max) to (Omega);
}\end{center}
\end{minipage}
\end{tabular}\end{center}

\noindent \textbf{ Теорема. }
$\Gamma(\mathcal{A})$ --- гёделева алгебра.

\noindent \textbf{ Теорема. }
Рассмотрим оценку $\llbracket \alpha \rrbracket_{\Gamma(\mathcal{L})} = \llbracket \alpha \rrbracket_\mathcal{L}$.
Тогда она является согласованной с ИИВ.

\subsection{Дизъюнктивность ИИВ (формулировка).}
\noindent \textbf{ Определение. }
Исчисление дизъюнктивно, если при любых $\alpha$ и $\beta$ из $\vdash\alpha\vee\beta$ следует $\vdash\alpha$ или $\vdash\beta$.

\noindent \textbf{ Определение. }
Решётка гёделева, если $a + b = 1$ влечёт $a = 1$ или $b = 1$.

\noindent \textbf{ Теорема. }
Интуиционистское исчисление высказываний дизъюнктивно.

\section{Разрешимость интуиционистского исчисления высказываний (формулировка).}
\noindent \textbf{ Теорема. }
Если $\not\vdash \alpha$ в ИИВ, то существует $\mathcal{G}$,
что $\mathcal{G} \not\models \alpha$, причём $|\mathcal{G}| \le 2^{2^{|\alpha|+2}}$.

\noindent \textbf{ Теорема. }
ИИВ разрешимо.


\section{Исчисление предикатов.}
\subsection{Язык исчисления предикатов.}
\noindent \textbf{ Определение. }
\begin{enumerate}
\item Два типа: предметные и логические выражения. 
\item Предметные выражения: метапеременная {\color{blue}$\theta$}. 
\begin{enumerate}
\item Предметные переменные: {\color{blue}$a$}, {\color{blue}$b$}, {\color{blue}$c$}, \dots, метапеременные {\color{blue}$x$}, {\color{blue}$y$}. 
\item Функциональные выражения: {\color{blue}$f(\theta_1,\dots,\theta_n)$}, метапеременные {\color{blue}$f$}, {\color{blue}$g$}, \dots\\
%Имена разнообразны: {\color{blue}$(\theta_1+\theta_2)$}, {\color{blue}$0$} и т.п.
\item Примеры: %{\color{blue}$(x+1)^2$}. \\
% Раскроем сокращения: {\color{blue}$(\theta_0+\theta_1)\equiv p(\theta_0,\theta_1)$},
%   {\color{blue}$\theta_1^2\equiv q(\theta_1)$}, {\color{blue}$1\equiv r$}, {\color{blue}$2\equiv s$}. \\
  {\color{blue}$r$},
  {\color{blue}$q(p(x,s),r)$}.
\end{enumerate}
\item Логические выражения: метапеременные {\color{blue}$\alpha$}, {\color{blue}$\beta$}, {\color{blue}$\gamma$}, \dots
\begin{enumerate}
\item Предикатные выражения: {\color{blue}$P(\theta_1,\dots,\theta_n)$}, метапеременная {\color{blue}$P$}.\\
Имена: {\color{blue}$A$}, {\color{blue}$B$}, {\color{blue}$C$}, \dots  %также {\color{blue}$(\theta_1=\theta_2)$} и т.п.
\item Связки: {\color{blue}$(\varphi\vee\psi)$}, {\color{blue}$(\varphi\with\psi)$}, {\color{blue}$(\varphi\rightarrow\psi)$}, 
   {\color{blue}$(\neg\varphi)$}.
\item Кванторы: {\color{blue}$(\forall x.\varphi)$} и {\color{blue}$(\exists x.\varphi)$}.
\end{enumerate}
\end{enumerate}
\subsection{Сокращения метаязыка для исчисления предикатов.}

\noindent \textbf{ Определение. }
\begin{enumerate}
  \item Метапеременные:
  \begin{enumerate}
  \item $\color{blue}\psi$, $\color{blue}\phi$, $\color{blue}\pi$, \dots --- формулы
  \item $\color{blue}P$, $\color{blue}Q$, \dots --- предикатные символы
  \item $\color{blue}\theta$, \dots --- термы
  \item $\color{blue}f$, $\color{blue}g$, \dots --- функциональные символы
  \item $\color{blue}x$, $\color{blue}y$, \dots --- предметные переменные
  \end{enumerate}
  
  \item Скобки --- как в И.В.; квантор --- жадный:
  \begin{center}${\color{blue}(\forall a.} \underbrace{{\color{blue}A \vee B \vee C \rightarrow \exists b.}
                      \underbrace{\color{blue}D \with \neg E}_{\exists b.\dots}}_{\forall a.\dots} \color{blue}) \with F$\end{center}
  
  
  \item Дополнительные обозначения при необходимости:
  \begin{enumerate}
  \item $\color{blue}(\theta_1 = \theta_2)$ вместо $\color{blue}E(\theta_1,\theta_2)$
  \item $\color{blue}(\theta_1 + \theta_2)$ вместо $\color{blue}p(\theta_1,\theta_2)$
  \item $\color{blue}0$ вместо $\color{blue}z$
  \item \dots
  \end{enumerate}
  
  \end{enumerate}

\subsection{Cледование в исчислении предикатов.}
\noindent \textbf{ Определение. }
$\gamma_1,\gamma_2,\dots,\gamma_n\models\alpha$, если выполнено два условия:
\begin{enumerate}
\item $\alpha$ выполнено всегда, когда выполнено $\gamma_1,\gamma_2,\dots,\gamma_n$;
\item $\alpha$ не использует кванторов по переменным, входящим свободно в $\gamma_1,\gamma_2,\dots,\gamma_n$.
\end{enumerate}

\noindent \textbf{ Теорема. }
Если $\Gamma\vdash\alpha$ и в доказательстве не используются кванторы по свободным
переменным из $\Gamma$, то $\Gamma\models\alpha$\
\subsection{Теорема о дедукции в исчислении предикатов (формулировка).}
\noindent \textbf{ Теорема. }
Если $\Gamma\vdash\alpha\rightarrow\beta$, то $\Gamma,\alpha\vdash\beta$.
Если $\Gamma,\alpha\vdash\beta$ и в доказательстве не применяются правила для кванторов 
по свободным переменным из $\alpha$, то $\Gamma\vdash\alpha\rightarrow\beta$.

\subsection{Теорема о корректности исчисления предикатов (формулировка).}
\noindent \textbf{ Теорема. }
Если $\theta$ свободен для подстановки 
вместо $x$ в $\varphi$, то $\llbracket\varphi\rrbracket^{x := \llbracket\theta\rrbracket} = \llbracket\varphi[x := \theta]\rrbracket$

\noindent \textbf{ Теорема. }
Если $\Gamma \vdash \alpha$ и в доказательстве не используются кванторы по свободным переменным из $FV(\Gamma)$, то $\Gamma \models \alpha$\
\end{document}
